% Javier initial version
%
% HII regions are far from been static objects.  Initially, the gas of
% a HII region is ionized in a previously disturbed medium.  When the
% recently created star starts emitting, the energy and matter coming
% from it interacts with the surrounding medium, putting in motion a
% series of dynamical processes that have a considerable effect in the
% velocity field of the now HII region.  Different investigations have
% shown that a complex velocity structure is particularly to each
% region.  The excess in the thermal broadening on the spectral lines
% is interpreted as disorder motions along the line-of-sight (LOS).
% Large-scale motions like the expansion or rotation of the region can
% be accounted for, but even after these are considered still persist
% a random component in the region velocity field.  It has become
% customary to investigate this random component as turbulence in the
% ionized gas.

% Javier's original para 2

% The fluctuations associated with the centroid velocities on the plane-of-sky (POS) of photoionized regions have been characterized several times using the second-order structure function.
% This studies span from galactic HII regions \citep{1986ApJ...300..624R} like Orion \citep{von1951methode,munch1958internal,castaneda1988,1992ApJ...387..229O,arthur2016turbulence} to giant HII regions \citep{1961MNRAS.122....1F,lagrois2009multi,lagrois2011}, like NGC 604 in the M 33 galaxy \citep{tanco1997,2019arXiv191203543M}.

% The procedure using the structure function allow us to determine a distinction between homogeneous turbulence and pure random velocity fluctuations in the photoionized gas.
% Several problems arise in the study of these fluctuation.
% One of them is that there is no well-established theory for turbulence in astronomical conditions; these being that the medium is compressible and supersonic.
% This conditions will develop shock waves in the HII region changing the energy balance \citep{lagrois2011,arthur2016turbulence}.
% And three, in regard with observations, there is the projection of a three-dimensional function into the plane-of-sky \citep{von1951methode,munch1958internal,1964SvA.....8..210K} leading to a loss in information of the velocity spectrum.

% There are also a series of problems related to the observations that to date have not been formally addressed like the previous ones.
% These problems are that: 1) following observations, there is the seeing and noise due to the telescope and instrument, and 2) that the observations are mostly limited to only one part of the nebula.
% This has the consequence that when analyzing a turbulent field, the velocity fluctuations related to small and large scales will be affected due to artifacts introduced by the problems mentioned above.
% In turbulence theory, the largest and smallest scales are the most important as they relate to the energy transfer of the turbulence spectrum, which implies that it is necessary to consider them for a correct characterization of the turbulence in the velocity field.
% Disregarding the above has led previous investigations in HII regions to erroneous conclusions about the nature of the energy spectrum.


%The idea of a virial system has been developed by \citet{1993ApJ...418..767T,munoz1996} into the \textit{Cometary Stirring Model} (CSM) with the aim to separate the mechanism that contribute to the line broadening.
%The stellar winds also tend to be used for an explanation of the line broadening and turbulence in GEHRs \citep{1994ApJ...425..720C}.
%This features seems to drive visually the regions with filaments, arcs and shells.
%On their spectra double and multiple emission features are signature of these objects.
%\citet{2020MNRAS.494...97S} argue that winds on ionizing clusters in GEHR must merge into a hot cluster because of the close proximity of the massive stars.
%\citet{2019ApJ...871...17U} propose that stellar feedback is an important source of energy to maintain turbulence in nearby galaxies.


%Other statistical techniques like velocity channel analysis (VCA) \citep{2000ApJ...537..720L} have been proved to better recreate the spectrum of the velocity field \citep{medina2014,arthur2016turbulence}.
%More recently, the structure function was used with full 6D measurements (position-velocity) to study the motions of stars in the Orion Molecular Cloud Complex.
%\citet{2021ApJ...907L..40H} argue that newborn stars should reflect the turbulent kinematics of their natal clouds.
%Using this previous method they are free of the projected element that interfere with previous gas-based studies and their results agree with the scaling law proposed by \citet{1981MNRAS.194..809L}.

%It is known that the speed of sound in a HII region is of the order of \(c_{HII} \sim\) 13 km s\kms.
%\citet{smith1970} discovered that the probable velocities of internal motions of extragalactic HII regions, \(\sigma_{W}\) (line width integrated over the whole HII region), were in the range of $\sim$ 19-34 km s\kms\ covering supersonic values.
%For the velocity to be supersonic it is necessary a continuous mechanism to maintain the observed motions, as supersonic motions are expected to dissipate energy via shocks \citep{1994Ap&SS.216..285C}.
%Turbulence is suggested as a mechanism that can stir the medium and maintain the supersonic velocities.



%\subsection{What turbulence scales can tell us about kinematics of the ionized gas?}

%In the Table \ref{tab:ResII} the second column show the separation where the value 2\(\sigma\) is reached.
%We also show the ratio \(r_{max}/r_{0}\), where \(r_{max}\) is the maximum separation or lag in pc; this value is related to the size of each sample box. 
%The forth column shows the Taylor scale, \(l_T\), which measures the average spatial extent of velocity gradients. \citet{1999ApJ...524..895M} define this scale as:

%\begin{equation}\label{eq:TS}
%l_T=\dfrac{\sigma r_s}{\sigma_{r_s}}
%\end{equation}

%where \(r_s\) is the smallest separation \(r\) and \(\sigma_{r_s}\) the value of the structure function at that separation. 

%It is possible to calculate how this Taylor micro-scale is related to the correlation length by:

%\begin{equation}\label{eq:TS1}
%l_T=\dfrac{r_s}{\sqrt{2}}(\dfrac{r_0}{r_s})^{m/2}
%\end{equation}  

%Finally in the table we present the Reynolds Number...

%\begin{table}
%\begin{center}\caption{Other significant turbulent scales}
%\begin{tabular}{cccccc}\hline
%HII         &r$_{2\sigma}$ &\(r_{max}\)/\(r_0\)  &\(l_T\) & \(Re\)* \\
%Region      &[pc]          &                     &[pc]    &   \\ \hline
%NGC 604     &28.73         &15.49                &5.35    &   \\
%NGC 595     &18.50         &21.01                &5.57    &   \\
%Hubble V    &11.61         &19.95                &1.80    &   \\ 
%Hubble X    &21.99         &21.67                &1.97    &   \\   
%30 Dor      &11.26         &8.99                 &0.59    &   \\
%Carina      &1.93          &31.55                &0.402   &   \\
%NGC 346     &5.78          &10.73                &0.39    &   \\
%Lagoon      &6.14          &10.68                &0.66    &   \\ 
%Orion Large &-             &2.98                 &0.24    &   \\
%Orion Small &0.162         &10.42                &0.019   &   \\ \hline	  
%\end{tabular}\label{tab:ResII}
%\end{center}
%\end{table}  

%With the aim to relate each \(r_0\) with each region we invoke previous studies on the kinematics of the ionized gas. 

%For the NGC 604 case, there is no clear relation between the \(r_0\) and some particular energy injection mechanism...
%it is also the nearest analog of more extreme star-forming regions, such as 30 Doradus in the Large Magellanic Cloud.

%%%%%%%%%%%%%%%other

The groundwork for the statistical study of turbulence was laid by \citet{taylor1935i,taylor1935ii} and \citet{karman1937statistical} and later developed by \citet{kolm1,kolm2} and \citet{heisenberg1951stability}. 
These mathematical techniques were first applied to astronomical observations by \citet{von1951methode} and \citet{munch1958internal} allowing to determine a distinction between homogeneous turbulence and pure random velocity fluctuations in the Orion Nebula. 

%%%%%%%%%%%%%%%%%%%%%%%%%%%%%%%%%%%%%%%%%%%%%%%%%%%%%%%%%%%%%%%%%%%%%%%%%%%%%%%%%%%%%%%%%%%%%%%%%%%%%%

%\citet{1983A&A...119..185V} made an estimate of the total masses of HI and HII in the nebula to obtain values of M$_{HI}=$ 1.2 $\times$ 10 $^{6}$ M$_{\odot}$ and M$_{HII} =$ 4.6 $\times$ 10 $^{5}$ M$_{\odot}$, and a ionizing luminosity of the star cluster is estimated at 5 $\times$ 10$^{50}$ erg s $^{1}$.
%The electronic temperature is 7,670 $\pm$116 K \citep{2010MNRAS.402.1635R}.
%2013ApJ...773...69G M33 dist
% \citep{1984ApJ...287..116K} \citep{1986ApJ...300..624R}

%The complexity of the extraction of pure turbulent motions from velocity profiles in nebulae is not just an observational problem, there is also  \citep{2011MNRAS.413..705L,arthur2016turbulence}. 

%We can compute the \(m_{3D}\) index from the two dimensional \(m\), if we assume that the \(m_{3D}\) index lies between the limits of projection smoothing and a sheet-like distribution. 
%Following the procedure by \cite{arthur2016turbulence} and assuming the relationship \(m_{2D} - 1 < m_{3D}\ < m_{2D}\) we have 0.54 \(< m_{3D} <\) 0.72.


%There is a good agreement with the single-Gaussian fit results from \citet{2019arXiv191203543M}.

%For all regions except for Carina, Lagoon and the EON there is an increase in this value.
%For 30 Doradus, which has the highest value, \(\sigma\) goes from \SI{16}{km.s^{-1}} to  \SI{19}{km.s^{-1}}.
%This is two times higher than NGC 604 that has a change of \SI{7}{km.s^{-1}} to \SI{9}{km.s^{-1}}, followed by NGC 595 with \SI{6.6}{km.s^{-1}} to \SI{7.5}{km.s^{-1}}.
%NGC 346 is next with a value of \SI{6}{km.s^{-1}} which is close to the value of the centroid velocities.
%Carina presents a value of \SI{4.2}{km.s^{-1}} which is the same as the velocity sample.
%The Orion core has a change from \SI{3}{km.s^{-1}} to \SI{4}{km.s^{-1}}.
%Hubble X has a value of \SI{3.9}{km.s^{-1}} which is very close with respect the \SI{3.6}{km.s^{-1}} value of the velocity field.
%The value for Hubble V increase from \SI{2.8}{km.s^{-1}} to \SI{3.3}{km.s^{-1}}.
%Lagoon presents a value of \SI{2.6}{km.s^{-1}} and the EON value decrease from \SI{3.2}{km.s^{-1}} to \SI{2.2}{km.s^{-1}}.

%It is known that HII regions are created on the edge of molecular clouds, so they will reach a blister or champagne phase, introducing large-scale gradients \citep{Mivi1995}. 
%Also there is the expansion or rotation of each region that would affect the large-scale motions.

%In this way we want to avoid one fundamental problem in the interpretation of the structure function. 
%The slope in the majority of previous studies and hence, the correlation length, is obtained without a formal procedure. 
%The traditional way to find the correlation length it is to determine a particular range of ascending values in the structure function. 
%If this increase in values is stopped or reversed it is commonly assumed that one particular inertial range has been found. 
%In some cases this procedure keeps going up to large scales, giving two or more inertial regimes leading to an erroneous interpretation of multiple inertial scales.

%%%%%%%%%%%%%%%%%

%\begin{enumerate}
%\item For 30 Doradus with an observational box size of 32 pc an index of 0.84$\pm$0.03 is obtained with an inertial scale between 0.1 and 5 pc, and with a variance of 350$\pm$21 km$^2$/s$^2$. 
    
%\item For NGC 604 with an observational box size of 173 pc an index of 0.82$\pm$0.20 is obtained with an inertial scale between 2 and 9 pc, and with a variance of 81$\pm$25 km$^2$/s$^2$.
    
%\item For NGC 595 with an observational box size of 196 pc an index of 1.3$\pm$0.04 is obtained with an inertial scale between 0.5 and 12 pc, and with a variance of 56$\pm$1.6 km$^2$/s$^2$.

%\item For NGC 346 with an observational box size of 20 pc an index of 0.78$\pm$0.02 is obtained with an inertial scale between 0.06 and 2 pc, and with a variance of 38$\pm$1.0 km$^2$/s$^2$.

%\item For Orion with an observational box size of 0.45 pc an index of 1.06$\pm$0.02 is obtained with an inertial scale between 0.002 and 0.09 pc, and with a variance of 17$\pm$1.0 km$^2$/s$^2$.

%\item For Carina with an observational box size of 20 pc an index of 1.36$\pm$0.30 is obtained with an inertial scale between 0.008 and 0.6 pc, and with a variance of 17$\pm$1.0 km$^2$/s$^2$.

%\item For Hubble X with an observational box size of 78 pc an index of 0.94$\pm$0.12 is obtained with an inertial scale between 0.5 and 4 pc, and with a variance of 15.5$\pm$1.5 km$^2$/s$^2$.

%\item For Hubble V with an observational box size of 61 pc an index of 0.72$\pm$0.09 is obtained with an inertial scale between 0.5 and 3 pc, and with a variance of 10.6$\pm$0.8 km$^2$/s$^2$.

%\item For Lagoon with an observational box size of 16 pc an index of 1.12$\pm$0.09 is obtained with an inertial scale between 0.005 and 1 pc, and with a variance of 7.2$\pm$0.8 km$^2$/s$^2$.

%\end{enumerate}

The index \(m\) shown in the last column of Table \ref{tab:Res} is used for comparison between observations and the theoretical value of 0.67 (equation \ref{eq:velocity-moments}).
All our indices are above this theoretical value and this is attributed to the fact that the \hii{} regions do not comply with the assumptions established in the development of the turbulent theory.

From section~\ref{sec:regions-milky-way} we know that most of all assumption mentioned in section~\ref{sec:turbulence-theory} are not meet for our \hii{} region sample.
\hii{} regions emerged from a molecular cloud with highly filamentary and clumpy density structure .
There is a large increase in the temperature because of the photoionization, and the density gradients with the pressure gradients accelerate the flow which occurs on multiple scales consequence of the fractal nature of the molecular cloud \citep{arthur2016turbulence}.
The radiation pressure, stellar winds and region expansion, among other physical processes would create pressure-driven shock waves propagating faster than the speed of sound in the medium.
This would make the flow compressible and supersonic in \hii{} regions.
Also, as \hii{} regions are three-dimensional objects our structure function results comes from a projection of this three-dimensional structure to a two-dimensional velocity field.
But despite the complex nature of \hii{} regions and observational limitations we obtain a common pattern for the structure function and the value of the power-law index falls in a certain limited range of values. 
The differences with respect the theoretical value are often used as a diagnostic tools to identify properties that are not consider in the assumptions of the turbulence theory.


%Deviations from the Kolmogorov model are expected in \hii{} regions as the photoionized flow in their interior has a different nature as the flows upon the theory is constructed. 
%Also magnetic fields would play a very important role in the dynamics of the gas motions.
%Despite these differences is clear that the self-similarity motions is a common behavior conserved at scales from parsecs to hundreds of parsecs. 
%With the advanced of the observational capabilities with more computational power for a better understanding of turbulent motions the gap between theory and observations is hoped to be reduced.

%%%%
%\subsection{Turbulence}\label{sec:turbulence-theory}

%\textit{Will 2021-12-10: Moving this section to later for now. Maybe
%  this material could go in discussion.}

%The idea to treat the ionized regions from a fluid mechanic view is intuitive if we consider the continuum hypothesis. 
%We validate this hypothesis using the Knudsen number defined as \(\text{Kn} \equiv \lambda / l\), where \(\lambda\) is the mean free path of the medium and \(l\) a characteristic length of the phenomenon we are studying.
%Taking into account a \(\lambda_{\text{HII}}\) of \SI{2.7e-7}{pc} \citep{1941ApJ....93..369S} and a \(l_{\text{HII}}\) of \SI{1}{pc} for ionized regions we obtained \(\text{Kn} \approx 10^{-7}\), while in general the continuum approach is valid for \(\text{Kn}\ll 1\).

%A turbulent behavior in the flows is more common in fluid systems than a laminar one.
%These turbulent flows tends to be unsteady, irregular and chaotic, and the turbulent motions are observed in many scales. 
%The turbulent flows are associated with a high Reynolds numbers which is defined as \(\text{Re} = l v_{l} / \nu \), where \(v_{l}\) is a characteristic velocity in the fluid system and \(\nu\) is the kinematic viscosity of the medium.
%Experiments have shown that a flow which is laminar become turbulent when increasing the velocity (i.e., increasing the Re) and a large range of velocity fluctuation appear at all scales.
%A high Re reflects high ratio of the advection/difussion terms of the Navier-Stokes equations, being interpreted as that the inertial forces taking over the viscous ones. 
%Hence it is possible to assume that if viscosity is not dominant in a flow, this flow would be turbulent.

%As astronomical observation made it possible to measure a large number of points with a high accuracy, it was found that velocities show a non-thermal component reflecting characteristic of turbulent earthly flows.
%The existence of turbulent flows in astrophysical objects, ranging in scales from 10\(^6\) to 10\(^{17}\) m \citep{2010ApJ...710..853C} is thought to be consequence of the high \(Re\), between 10\(^4\) and 10\(^9\) \citep{1949ApJ...110..329C,lagrois2011} that characterized these objects.

%The classic turbulence theory assumes a three-dimensional isotropic, incompressible and subsonic flow that can be described using an energy spectrum formed by vortices or eddies of different sizes \citep{kolm1}.
%The concept of the energy spectrum assumes that the energy is cascading down inside the flow from the largest to the smallest scales.
%The different scales in the energy cascade are defined based on their Re.
%In one side of the spectrum there is the energy injection scale, \(L_I\), that is related to the largest vortices, called the integral scales, where \(\text{Re} \rightarrow \infty\).
%On the other side of the spectrum there is the dissipation of energy at small scales, \(l_I\), where  \(\text{Re} \rightarrow 1\) and it is assumed that small vortices are responsible for the friction that dissipates the energy flux into thermal energy.
%In between there is the inertial range, \(l_d  << l << L_I\), where in a stationary case the energy transfer is at a constant rate.

%The rate of transfer of specific energy \(\epsilon\) (energy per unit mass) in the inertial range is independent of the \(l\) scale having the form \(\epsilon \sim v_l^2 / \tau_l\), where the kinetic energy is of the order of \( v_l^2\) and the life time of a velocity fluctuation is \(\tau_l\), which can be can be put in terms of \(\tau_l \sim l / v_l\). 
%Hence we obtain \( v_l \sim (\epsilon l)^{1/3}\) where \(\epsilon\) is invariant in the cascade.

%The energy cascade is described using the energy spectrum, $E(k) \propto k^{\beta}$ where \(k\) is the wave number for a determined scale \(l\) ($k \equiv l^{-1}$).
%The total specific energy of a determined $l$ scale is of the order \(\langle v_{l}^{2} \rangle \) and is the energy integral above the associated \(k\), defined as:

%\begin{equation}\label{eq:energy-spectrum}
% \langle v_{l}^{2} \rangle = \int_{k}^{\infty} E(k)dk
%\end{equation}
%
%If it is assumed that this energy is of the order of $(\epsilon l)^{\frac{2}{3}} \sim (\frac{\epsilon}{k})^{\frac{2}{3}}$ we can differentiate equation \ref{eq:energy-spectrum} and obtained the final form:

%\begin{equation}\label{eq:kolmogorov-law}
%E(k) \propto \epsilon^\frac{2}{3} k^{-\frac{5}{3}}
%\end{equation}
%
%where \(\beta = -5 / 3\) is the Kolmogorov law.
%For supersonic turbulence the index is $\beta=-2$ \citep{burg}. 

%Using a statistical law that relates the moments \(M_n\) of order \(n\) with the increments in the velocity field for a separation \(\boldsymbol{r}\) \citep{Leqism}, we have:

%\begin{equation}\label{eq:velocity-moments}
%M_n = \langle \vert \boldsymbol{v}(\boldsymbol{x} + \boldsymbol{r})- \boldsymbol{v}(\boldsymbol{x})\vert^n \rangle \propto (\epsilon r)^{n/3}
%\end{equation}
%
%the average is over all the positions \(\boldsymbol{x}\) and directions \(\boldsymbol{r}\). 
%The Kolmogorov model predicts that the moment of order 2, the second-order structure function (equation \ref{eq:Br}), follows \(M_2 \propto r^{2/3}\).

%\subsection{Origins of interstellar turbulence}\label{sec:origins-turbulence}

%In the particular case of GEHR, as supersonic motions dissipate energy by shocks in time scales relatively short compared with the age of the region, it is necessary an energy source to replenish continuously the energy lost. From the observed velocity  dispersion \(\sigma\), and assuming that we have an isotropic distribution of velocities,  we can estimate the 3D velocity dispersion by \(\sigma_\mathrm{3D} = \sqrt{3} \sigma\). The mass of the ionized gas together with the velocity dispersion \(\sigma_\mathrm{3D}\)  are used to estimate the kinetic energy in the GEHRs.
%As these objects have typically  masses of the ionized gas of \num{\sim e5} M\(_{\sun}\) and velocity dispersion values  of \SI{\sim 20}{km.s^{-1}} \citep{1984ApJ...287..116K}, typical kinetic energy of the ionized gas is \SI{\sim e51}{ergs}. 

%In the standard Kolmogorov model, the energy spectrum of turbulence becomes dependent only on \(\epsilon\), the mean dissipation of energy per unit time per unit mass of fluid,  with the inertial forces transferring without losses kinetic energy from larger to smaller turbulent elements within the inertial range. Let \(L_c\) be the characteristic length of the largest eddies, and \(\delta v\) the typical velocity. 

%We have then that \(\epsilon = (\delta v)^{3}/L\). In steady-state turbulence, equilibrium exists between the energy input and the energy dissipation rates. 
%Although the present observations indicate that turbulence in \hii{} regions may not follow the Kolmogorov model, with energy input occurring at different scales, so that a  truly inertial range may not exist at all, lets us make a crude estimation of the  amount of energy transferred between scales using this simple model. 
%Assuming the above figures for the mass of the gas and velocity dispersion, \(\delta v = \sigma_\mathrm{3D}\),  and a \(L_c\) of \SI{100}{pc}, we obtain that the energy transfer rate \SI{\sim e37}{erg.s^{-1}}, and  the time scale for energy transfer, \(L/\delta v \), would be of few million years.

%There are several energy sources that could provide or contribute  to the energy required: of the order of kpc, the differential rotations of the galaxy that produces shear at large scale, at scales of the order of hundreds of pc, supernova explosions inject a large amount of kinetic energy.
%At scales of several tens of pc the expansion of \hii{} regions and stellar winds are a source of energy and, bipolar flows, shocks, jets and champagne flows inject energy at small scales about one tenth of a pc.

%The lack of precise information on how these mechanism contribute to the emission line broadening make a non-trivial task to determine at what extent they play a role in the interstellar turbulence, contributing to the complexity of the extraction of pure turbulent motions from velocity profiles in nebulae \citep{2011MNRAS.413..705L,arthur2016turbulence}. 