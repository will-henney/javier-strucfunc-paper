% Javier initial version
%
% HII regions are far from been static objects.  Initially, the gas of
% a HII region is ionized in a previously disturbed medium.  When the
% recently created star starts emitting, the energy and matter coming
% from it interacts with the surrounding medium, putting in motion a
% series of dynamical processes that have a considerable effect in the
% velocity field of the now HII region.  Different investigations have
% shown that a complex velocity structure is particularly to each
% region.  The excess in the thermal broadening on the spectral lines
% is interpreted as disorder motions along the line-of-sight (LOS).
% Large-scale motions like the expansion or rotation of the region can
% be accounted for, but even after these are considered still persist
% a random component in the region velocity field.  It has become
% customary to investigate this random component as turbulence in the
% ionized gas.

% Javier's original para 2

% The fluctuations associated with the centroid velocities on the plane-of-sky (POS) of photoionized regions have been characterized several times using the second-order structure function.
% This studies span from galactic HII regions \citep{1986ApJ...300..624R} like Orion \citep{von1951methode,munch1958internal,castaneda1988,1992ApJ...387..229O,arthur2016turbulence} to giant HII regions \citep{1961MNRAS.122....1F,lagrois2009multi,lagrois2011}, like NGC 604 in the M 33 galaxy \citep{tanco1997,2019arXiv191203543M}.

% The procedure using the structure function allow us to determine a distinction between homogeneous turbulence and pure random velocity fluctuations in the photoionized gas.
% Several problems arise in the study of these fluctuation.
% One of them is that there is no well-established theory for turbulence in astronomical conditions; these being that the medium is compressible and supersonic.
% This conditions will develop shock waves in the HII region changing the energy balance \citep{lagrois2011,arthur2016turbulence}.
% And three, in regard with observations, there is the projection of a three-dimensional function into the plane-of-sky \citep{von1951methode,munch1958internal,1964SvA.....8..210K} leading to a loss in information of the velocity spectrum.

% There are also a series of problems related to the observations that to date have not been formally addressed like the previous ones.
% These problems are that: 1) following observations, there is the seeing and noise due to the telescope and instrument, and 2) that the observations are mostly limited to only one part of the nebula.
% This has the consequence that when analyzing a turbulent field, the velocity fluctuations related to small and large scales will be affected due to artifacts introduced by the problems mentioned above.
% In turbulence theory, the largest and smallest scales are the most important as they relate to the energy transfer of the turbulence spectrum, which implies that it is necessary to consider them for a correct characterization of the turbulence in the velocity field.
% Disregarding the above has led previous investigations in HII regions to erroneous conclusions about the nature of the energy spectrum.


%The idea of a virial system has been developed by \citet{1993ApJ...418..767T,munoz1996} into the \textit{Cometary Stirring Model} (CSM) with the aim to separate the mechanism that contribute to the line broadening.
%The stellar winds also tend to be used for an explanation of the line broadening and turbulence in GEHRs \citep{1994ApJ...425..720C}.
%This features seems to drive visually the regions with filaments, arcs and shells.
%On their spectra double and multiple emission features are signature of these objects.
%\citet{2020MNRAS.494...97S} argue that winds on ionizing clusters in GEHR must merge into a hot cluster because of the close proximity of the massive stars.
%\citet{2019ApJ...871...17U} propose that stellar feedback is an important source of energy to maintain turbulence in nearby galaxies.


%Other statistical techniques like velocity channel analysis (VCA) \citep{2000ApJ...537..720L} have been proved to better recreate the spectrum of the velocity field \citep{medina2014,arthur2016turbulence}.
%More recently, the structure function was used with full 6D measurements (position-velocity) to study the motions of stars in the Orion Molecular Cloud Complex.
%\citet{2021ApJ...907L..40H} argue that newborn stars should reflect the turbulent kinematics of their natal clouds.
%Using this previous method they are free of the projected element that interfere with previous gas-based studies and their results agree with the scaling law proposed by \citet{1981MNRAS.194..809L}.

%It is known that the speed of sound in a HII region is of the order of \(c_{HII} \sim\) 13 km s\kms.
%\citet{smith1970} discovered that the probable velocities of internal motions of extragalactic HII regions, \(\sigma_{W}\) (line width integrated over the whole HII region), were in the range of $\sim$ 19-34 km s\kms\ covering supersonic values.
%For the velocity to be supersonic it is necessary a continuous mechanism to maintain the observed motions, as supersonic motions are expected to dissipate energy via shocks \citep{1994Ap&SS.216..285C}.
%Turbulence is suggested as a mechanism that can stir the medium and maintain the supersonic velocities.



%\subsection{What turbulence scales can tell us about kinematics of the ionized gas?}

%In the Table \ref{tab:ResII} the second column show the separation where the value 2\(\sigma\) is reached.
%We also show the ratio \(r_{max}/r_{0}\), where \(r_{max}\) is the maximum separation or lag in pc; this value is related to the size of each sample box. 
%The forth column shows the Taylor scale, \(l_T\), which measures the average spatial extent of velocity gradients. \citet{1999ApJ...524..895M} define this scale as:

%\begin{equation}\label{eq:TS}
%l_T=\dfrac{\sigma r_s}{\sigma_{r_s}}
%\end{equation}

%where \(r_s\) is the smallest separation \(r\) and \(\sigma_{r_s}\) the value of the structure function at that separation. 

%It is possible to calculate how this Taylor micro-scale is related to the correlation length by:

%\begin{equation}\label{eq:TS1}
%l_T=\dfrac{r_s}{\sqrt{2}}(\dfrac{r_0}{r_s})^{m/2}
%\end{equation}  

%Finally in the table we present the Reynolds Number...

%\begin{table}
%\begin{center}\caption{Other significant turbulent scales}
%\begin{tabular}{cccccc}\hline
%HII         &r$_{2\sigma}$ &\(r_{max}\)/\(r_0\)  &\(l_T\) & \(Re\)* \\
%Region      &[pc]          &                     &[pc]    &   \\ \hline
%NGC 604     &28.73         &15.49                &5.35    &   \\
%NGC 595     &18.50         &21.01                &5.57    &   \\
%Hubble V    &11.61         &19.95                &1.80    &   \\ 
%Hubble X    &21.99         &21.67                &1.97    &   \\   
%30 Dor      &11.26         &8.99                 &0.59    &   \\
%Carina      &1.93          &31.55                &0.402   &   \\
%NGC 346     &5.78          &10.73                &0.39    &   \\
%Lagoon      &6.14          &10.68                &0.66    &   \\ 
%Orion Large &-             &2.98                 &0.24    &   \\
%Orion Small &0.162         &10.42                &0.019   &   \\ \hline	  
%\end{tabular}\label{tab:ResII}
%\end{center}
%\end{table}  

%With the aim to relate each \(r_0\) with each region we invoke previous studies on the kinematics of the ionized gas. 

%For the NGC 604 case, there is no clear relation between the \(r_0\) and some particular energy injection mechanism...
%it is also the nearest analog of more extreme star-forming regions, such as 30 Doradus in the Large Magellanic Cloud.

%%%%%%%%%%%%%%%other

The groundwork for the statistical study of turbulence was laid by \citet{taylor1935i,taylor1935ii} and \citet{karman1937statistical} and later developed by \citet{kolm1,kolm2} and \citet{heisenberg1951stability}. 
These mathematical techniques were first applied to astronomical observations by \citet{von1951methode} and \citet{munch1958internal} allowing to determine a distinction between homogeneous turbulence and pure random velocity fluctuations in the Orion Nebula. 

%%%%%%%%%%%%%%%%%%%%%%%%%%%%%%%%%%%%%%%%%%%%%%%%%%%%%%%%%%%%%%%%%%%%%%%%%%%%%%%%%%%%%%%%%%%%%%%%%%%%%%

%\citet{1983A&A...119..185V} made an estimate of the total masses of HI and HII in the nebula to obtain values of M$_{HI}=$ 1.2 $\times$ 10 $^{6}$ M$_{\odot}$ and M$_{HII} =$ 4.6 $\times$ 10 $^{5}$ M$_{\odot}$, and a ionizing luminosity of the star cluster is estimated at 5 $\times$ 10$^{50}$ erg s $^{1}$.
%The electronic temperature is 7,670 $\pm$116 K \citep{2010MNRAS.402.1635R}.
%2013ApJ...773...69G M33 dist
% \citep{1984ApJ...287..116K} \citep{1986ApJ...300..624R}

%The complexity of the extraction of pure turbulent motions from velocity profiles in nebulae is not just an observational problem, there is also  \citep{2011MNRAS.413..705L,arthur2016turbulence}. 

%We can compute the \(m_{3D}\) index from the two dimensional \(m\), if we assume that the \(m_{3D}\) index lies between the limits of projection smoothing and a sheet-like distribution. 
%Following the procedure by \cite{arthur2016turbulence} and assuming the relationship \(m_{2D} - 1 < m_{3D}\ < m_{2D}\) we have 0.54 \(< m_{3D} <\) 0.72.


%There is a good agreement with the single-Gaussian fit results from \citet{2019arXiv191203543M}.
