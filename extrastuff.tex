We use data from \citet{1987A&A...176..347H} and \citep{arthur2016turbulence} to study of turbulence in the Orion Nebula.
\cite{1987A&A...176..347H} observations were obtained using a 106 cm-Cassegrain telescope at Observatorium Hoher List. 
A Fabry-Pérot interferometer was used having the étalon a separation of 0.5 mm. 
Interferograms have been taken with different pointing directions of the telescope's optical axis in the nebula. 
The exposures are overlapped and fall into one square of a grid with a width of 1' centered at \(\theta^{1}\)Ori C.   


%The idea of a virial system has been developed by \citet{1993ApJ...418..767T,munoz1996} into the \textit{Cometary Stirring Model} (CSM) with the aim to separate the mechanism that contribute to the line broadening.
%The stellar winds also tend to be used for an explanation of the line broadening and turbulence in GEHRs \citep{1994ApJ...425..720C}.
%This features seems to drive visually the regions with filaments, arcs and shells.
%On their spectra double and multiple emission features are signature of these objects.
%\citet{2020MNRAS.494...97S} argue that winds on ionizing clusters in GEHR must merge into a hot cluster because of the close proximity of the massive stars.
%\citet{2019ApJ...871...17U} propose that stellar feedback is an important source of energy to maintain turbulence in nearby galaxies.

%The classic turbulence theory \citep{kolm1} assumes a three-dimensional isotropic, incompressible and subsonic flow where an energy cascade is taking place.
%The interstellar matter does not have those physical properties and our observations are in two-dimensions.
%But as mentioned above the idea of an energy cascade is used to understand our observations.
%The energy cascade goes from the largest vortices where the energy in injected until it reaches dissipation with the smallest vortices.
%In the energy cascade three scales are defined based on their Reynolds number.
%The energy injection scale (Re $\rightarrow \infty$), $L_{EI}$, the dissipation scale (Re $\rightarrow$ 1), $L_{D}$ and the inertial scale, $L_{I}$, between the previous two.
%Since the energy distributions depends on the scale, the energy cascade concept is described with an energy spectrum, $E(k) \propto k$ where $k$ is the wave number for a determined scale $l$ ($k \equiv r^{-1}$).

%Other statistical techniques like velocity channel analysis (VCA) \citep{2000ApJ...537..720L} have been proved to better recreate the spectrum of the velocity field \citep{medina2014,arthur2016turbulence}.
%More recently, the structure function was used with full 6D measurements (position-velocity) to study the motions of stars in the Orion Molecular Cloud Complex.
%\citet{2021ApJ...907L..40H} argue that newborn stars should reflect the turbulent kinematics of their natal clouds.
%Using this previous method they are free of the projected element that interfere with previous gas-based studies and their results agree with the scaling law proposed by \citet{1981MNRAS.194..809L}.


%It is known that the speed of sound in a HII region is of the order of \(c_{HII} \sim\) 13 km s\kms.
%\citet{smith1970} discovered that the probable velocities of internal motions of extragalactic HII regions, \(\sigma_{W}\) (line width integrated over the whole HII region), were in the range of $\sim$ 19-34 km s\kms\ covering supersonic values.
%For the velocity to be supersonic it is necessary a continuous mechanism to maintain the observed motions, as supersonic motions are expected to dissipate energy via shocks \citep{1994Ap&SS.216..285C}.
%Turbulence is suggested as a mechanism that can stir the medium and maintain the supersonic velocities.
%The large lengths of astrophysical objects, ranging in scales from 10$^{6}$ to 10$^{17}$ m \citep{2010ApJ...710..853C}, and the high Reynolds numbers (Re), in the order of 10$^{4}$ - 10$^{9}$, are properties that supports the idea of a turbulent interstellar medium.


%The relationship between L vs \(\sigma\) is one of the most studied. 
%This relationship help us understanding how the star formation affects the gas motions and how star formation is regulated by feedback processes. 
%Figure \ref{fig:sigvsl} shows different studies of the L vs \(\sigma\) relationship \citep{1988A&A...201..199A,Rozas:2006b,2015MNRAS.449.3568M}. 
%In this case it is normally used the deconvolution of the observed lines taking account thermal broadening and instrumental broadening leaving the \(\sigma_{LOS}\). 
%For the data present in Figure \ref{fig:sigvsl}, excluding \(\sigma_{POS}\), we obtain using the same OLS \(log \sigma_{LOS} = 0.145 \pm 0.008 (logL_{H\alpha}) - (4.42 \pm 0.336) \).
%Since the R-squared is 0.596 it is not possible just to invert the OLS results for a comparison with the more classical relationship. 
%Inverting axes we obtain \(log L_{H\alpha} = 4.09 \pm 0.23 log (\sigma_{LOS}) + (34.17 \pm 0.32) \), being in accord to previous relationships. 


%\subsection{What turbulence scales can tell us about kinematics of the ionized gas?}

%In the Table \ref{tab:ResII} the second column show the separation where the value 2\(\sigma\) is reached.
%We also show the ratio \(r_{max}/r_{0}\), where \(r_{max}\) is the maximum separation or lag in pc; this value is related to the size of each sample box. 
%The forth column shows the Taylor scale, \(l_T\), which measures the average spatial extent of velocity gradients. \citet{1999ApJ...524..895M} define this scale as:

%\begin{equation}\label{eq:TS}
%l_T=\dfrac{\sigma r_s}{\sigma_{r_s}}
%\end{equation}

%where \(r_s\) is the smallest separation \(r\) and \(\sigma_{r_s}\) the value of the structure function at that separation. 

%It is possible to calculate how this Taylor micro-scale is related to the correlation length by:

%\begin{equation}\label{eq:TS1}
%l_T=\dfrac{r_s}{\sqrt{2}}(\dfrac{r_0}{r_s})^{m/2}
%\end{equation}  

%Finally in the table we present the Reynolds Number...

%\begin{table}
%\begin{center}\caption{Other significant turbulent scales}
%\begin{tabular}{cccccc}\hline
%HII         &r$_{2\sigma}$ &\(r_{max}\)/\(r_0\)  &\(l_T\) & \(Re\)* \\
%Region      &[pc]          &                     &[pc]    &   \\ \hline
%NGC 604     &28.73         &15.49                &5.35    &   \\
%NGC 595     &18.50         &21.01                &5.57    &   \\
%Hubble V    &11.61         &19.95                &1.80    &   \\ 
%Hubble X    &21.99         &21.67                &1.97    &   \\   
%30 Dor      &11.26         &8.99                 &0.59    &   \\
%Carina      &1.93          &31.55                &0.402   &   \\
%NGC 346     &5.78          &10.73                &0.39    &   \\
%Lagoon      &6.14          &10.68                &0.66    &   \\ 
%Orion Large &-             &2.98                 &0.24    &   \\
%Orion Small &0.162         &10.42                &0.019   &   \\ \hline	  
%\end{tabular}\label{tab:ResII}
%\end{center}
%\end{table}  

%With the aim to relate each \(r_0\) with each region we invoke previous studies on the kinematics of the ionized gas. 
%\citet{sabalisck1995supersonic} studied NGC 604 revealing that the kinematics of the region behave different considering %different scales.
%High emission knots on the region encompasses supersonic global velocity dispersion values of 14 $<$ \(\sigma_{LOS}\) $<$ 20 km s\(^{-1}\).
%This knots are well fitted with Gaussian profiles thus, defining the kinematic core \citep{munoz1996}.
%A characteristic radius for them is \(\sim\) 7 pc.
%\citet{yang1996} identified 5 expanding shells within NGC 604, and one of them (Shell 3 on their paper) fall into our observations.
%This shells is 40 pc in diameter with an expansion velocity above 50 km s\(^{-1}\).
%This scales are related to the presence of strong stellar winds caused by massive stars.
%For the NGC 604 case, there is no clear relation between the \(r_0\) and some particular energy injection mechanism...
%it is also the nearest analog of more extreme star-forming regions, such as 30 Doradus in the Large Magellanic Cloud.

%%%%%%%%%%%%%%%other

The groundwork for the statistical study of turbulence was laid by \citet{taylor1935i,taylor1935ii} and \citet{karman1937statistical} and later developed by \citet{kolm1,kolm2} and \citet{heisenberg1951stability}. 
These mathematical techniques were first applied to astronomical observations by \citet{von1951methode} and \citet{munch1958internal} allowing to determine a distinction between homogeneous turbulence and pure random velocity fluctuations in the Orion Nebula. 

\section{Turbulence}\label{sec:turb}

The classic turbulence theory \citep{kolm1} assumes a three-dimensional isotropic, incompressible and subsonic flow, where an energy cascade, with different $l$ sizes in an auto-similar configuration, is taking place. 
The energy cascades goes downward, from the large vortices until it reaches dissipation with the smallest vortices.
In this energy cascade three scales can be defined: the energy injection scale (Re $\rightarrow \infty$), $L_{EI}$, the dissipation scale (Re $\rightarrow$ 1), $L_{D}$ and the inertial scale ($L_{I}$) between the previous two. 

In the mathematical description of the Kolmogorov theory the energy transfer, $\epsilon$, in the inertial range can be denoted as $\epsilon \sim \frac{v_{l}^{2}}{\tau_{l}}$, where $v_{l}^{2}$ is the kinetic energy per mass unit and $\tau_{l}$ the time related with velocity fluctuations. 
Kolmogorov states the relationship \citep{Leqism}:

\begin{equation}\label{eq:velsca}
v_{l} \sim (\epsilon l)^{\frac{1}{3}}
\end{equation}

The energy cascade concept can  be described with an energy spectrum, $E(k) \propto k^{-\beta}$ where $k$ is the wave number for a scale $l$ ($k \equiv l^{-1}$). The total specific energy of a determined $l$ scale is of the order $\langle v_{l}^{2} \rangle$ and is the energy integral above the associated $k$, defined as:

\begin{equation}\label{eq:veles}
 \langle v_{l}^{2} \rangle = \int_{k}^{\infty} E(k)dk
\end{equation}

Considering that this energy is of the order of $(\epsilon l)^{\frac{2}{3}} \sim (\frac{\epsilon}{k})^{\frac{2}{3}}$ we can substitute in equation \ref{eq:veles} and obtain the Kolmogorov law:

\begin{equation}\label{eq:kolm}
E(k) \propto \epsilon^\frac{2}{3} k^{-\frac{5}{3}}
\end{equation}

For subsonic (Kolmogorov) turbulence $\beta= -5/3$, and for supersonic turbulence the index is $\beta=-2$ \citep{burg}. 

Equation \ref{eq:velsca} is related to the quadratic velocities differences by:

\begin{equation}\label{eq:sfi}
(\Delta v)^{2} \sim (\epsilon l)^{\frac{2}{3}}
\end{equation}

We show in the next section that the term $(\Delta v)^{2}$ can be defined as the structure function. The left term with index that we define as $\alpha= 2/3$ in equation \ref{eq:sfi} is directly related to the energy, the left term in equation \ref{eq:veles}. This index $\alpha$ serves as a reference for the Kolmogorov theory and it is used for direct comparison with observations.
%%%%%%%%%%%%%%%%%%%%%%%%%%%%%%%%%%%%%%%%%%%%%%%%%%%%%%%%%%%%%%%%%%%%%%%%%%%%%%%%%%%%%%%%%%%%%%%%%%%%%%%%
\section{Turbulence}\label{sec:turb}

The classic turbulence theory \citep{kolm1} assumes a three-dimensional isotropic, incompressible and subsonic flow, where an energy cascade, with different $l$ sizes in an auto-similar configuration, is taking place. 
The energy cascades goes downward, from the large vortices until it reaches dissipation with the smallest vortices.
In this energy cascade three scales can be defined: the energy injection scale (Re $\rightarrow \infty$), $L_{EI}$, the dissipation scale (Re $\rightarrow$ 1), $L_{D}$ and the inertial scale ($L_{I}$) between the previous two. The energy cascade concept is described with an energy spectrum, $E(k) \propto k^{-\beta}$ where $k$ is the wave number for a scale $l$ ($k \equiv l^{-1}$).Following the mathematical description of the Kolmogorov theory the mean specific energy transfer, $\langle \epsilon \rangle$, in the inertial range can be denoted as $\epsilon \sim \frac{v_{l}^{2}}{\tau_{l}}$, where $v_{l}^{2}$ is the kinetic energy per mass unit and $\tau_{l}$ the time related with velocity fluctuations. Since the total specific energy of a determined $l$ scale is of the order $\langle v_{l}^{2} \rangle$ and is the energy integral above the associated $k$, defined as:

\begin{equation}\label{eq:veles}
 \langle v_{l}^{2} \rangle = \int_{k}^{\infty} E(k)dk
\end{equation}

Considering that this energy is of the order of $(\epsilon l)^{\frac{2}{3}} \sim (\frac{\epsilon}{k})^{\frac{2}{3}}$ we can substitute in equation \ref{eq:veles} and obtain the Kolmogorov law:

\begin{equation}\label{eq:kolm}
E(k) \propto \epsilon^\frac{2}{3} k^{-\frac{5}{3}}
\end{equation}

Considering the quadratic velocities differences:

\begin{equation}\label{eq:sfi}
(\Delta v)^{2} \sim (\epsilon l)^{\frac{2}{3}}
\end{equation}

the term $(\Delta v)^{2}$ can be defined as the structure function. The left term with index that we define as $\alpha= 2/3$ in equation \ref{eq:sfi} is directly related to the energy, the left term in equation \ref{eq:veles}. This index $\alpha$ serves as a reference for the Kolmogorov theory and it is used for direct comparison with observations.

The techniques we used to study the radial velocity sample, $V_{r}(\boldsymbol{x})$, where $V_{r}(\boldsymbol{x})= V_{obs}(\boldsymbol{x})-\langle V_{obs}(\boldsymbol{x}) \rangle$ of a two dimensional field were the second order structure function, $S_{2}(l)$, and the auto correlation function, $R(l)$, defined as: 

\begin{equation}\label{eq:S}
S_{2}(\boldsymbol{l})=\dfrac{\sum[V_{r}(\boldsymbol{x}+\boldsymbol{l})-V_{r}(\boldsymbol{x}) ]^{2}}{N(\boldsymbol{l})}
\end{equation}

where $\sigma^{2}$ is the variance of the sample and N($\boldsymbol{l}$) the number of points at each separation.

%%%%%%%%%%%%%%%%%%%%%%%%%%%%%%%%%%%%%%%%%%%%%%%%%%%%%%%%%%%%%%%%%%%%%%%%%%%

%The analysis employed until now is not conclusive about the true velocity fluctuations observed in each nebula. 
%As pointed out by \citet{arthur2016turbulence} other techniques also allow us to study these fluctuations. 


%\begin{table*}
% \begin{center}\caption{Main results.
%  a) \citet{tanco1997}
%      b) \citet{2019arXiv191203543M}
%      c) \citet{Castro:2018a}
%      d) \citet{Damiani:2016a}
%      e) \citet{Damiani:2017b}
%      f) \citet{1987A&A...176..347H}
%      g) \citet{arthur2016turbulence}.}
% \begin{tabular}{ccccccccc}\hline
% HII         &\(\sigma^{2}\) &\(r_0\)                     &\(m\)    &\(\langle \sigma_{LOS} \rangle \) & Previously\\
% Region      &[km/s]$^{2}$     &[pc]                        &                           &  [km/s]& used in:\\ \hline
% NGC 604     & 80.93$\pm$25.31 &  9.12$\pm$2.98  & 0.82$\pm$0.20  & 16.2 & a,b \\
% NGC 595     & 56.45$\pm$1.69  & 11.69$\pm$0.45  & 1.29$\pm$0.04  & 18.3 & - \\
% Hubble X    & 15.55$\pm$1.50  &  3.90$\pm$0.28  & 0.94$\pm$0.12  & 13.4 & - \\ 
% Hubble V    & 10.46$\pm$0.75  &  3.42$\pm$0.33  & 0.73$\pm$0.09  & 12.3 & - \\  
% 30 Dor      & 350.87$\pm$21.59 & 4.60$\pm$0.43  & 0.84$\pm$0.03  & 31.7 & c\\
% Carina      & 16.74$\pm$1.13  & 0.66$\pm$0.08  & 1.36$\pm$0.29  & 22.4 & d\\
% NGC 346     & 38.03$\pm$1.09  & 1.90$\pm$0.09  & 0.78$\pm$0.02  & 10.2 & -\\
% Lagoon      & 7.29$\pm$0.84   & 1.00$\pm$0.17  & 1.12$\pm$0.09  & 13.6 & e\\ 
% Orion Large & 5.97$\pm$0.58   & 0.73$\pm$0.17  & 1.72$\pm$0.524 & 6.0 & f \\
% Orion Small & 16.92$\pm$0.97  & 0.091$\pm$0.006 & 1.06$\pm$0.01 & 6.0 & g \\\hline	  
% \end{tabular}\label{tab:Res}
% \end{center}
%\end{table*}  