% mnras_template.tex 
%
% LaTeX template for creating an MNRAS paper
%
% v3.0 released 14 May 2015
% (version numbers match those of mnras.cls)
%
% Copyright (C) Royal Astronomical Society 2015
% Authors:
% Keith T. Smith (Royal Astronomical Society)

% Change log
%
% v3.0 May 2015
%    Renamed to match the new package name
%    Version number matches mnras.cls
%    A few minor tweaksto wording
% v1.0 September 2013
%    Beta testing only - never publicly released
%    First version: a simple (ish) template for creating an MNRAS paper

%%%%%%%%%%%%%%%%%%%%%%%%%%%%%%%%%%%%%%%%%%%%%%%%%%
% Basic setup. Most papers should leave these options alone.
\documentclass[fleqn,usenatbib, useAMS, a4paper]{mnras}

% MNRAS is set in Times font. If you don't have this installed (most LaTeX
% installations will e fine) or prefer the old Computer Modern fonts, comment
% out the following line
\usepackage{newtxtext}
\usepackage[varg,varvw,smallerops]{newtxmath}
% Depending on your LaTeX fonts installation, you might get better results with one of these:
%\usepackage{mathptmx}
%\usepackage{txfonts}

% Use vector fonts, so it zooms properly in on-screen viewing software
% Don't change these lines unless you know what you are doing
\usepackage[T1]{fontenc}
\usepackage{ae,aecompl}


%%%%% AUTHORS - PLACE YOUR OWN PACKAGES HERE %%%%%

% Only include extra packages if you really need them. Common packages are:
\usepackage{graphicx}	% Including figure files
\let\Bbbk\relax
\usepackage{amsmath}	% Advanced maths commands
\usepackage{amssymb}	% Extra maths symbols
\usepackage{multicol}

%%%%%%%%%%%%%%%%%%%%%%%%%%%%%%%%%%%%%%%%%%%%%%%%%%

%%%%% AUTHORS - PLACE YOUR OWN COMMANDS HERE %%%%%

% Please keep new commands to a minimum, and use \newcommand not \def to avoid
% overwriting existing commands. Example:
%\newcommand{\pcm}{\,cm$^{-2}$}	% per cm-squared

% A better \ion command that works in more circumstances
\newcommand\ION[2]{#1\,\scalebox{0.9}[0.8]{\uppercase{#2}}}

\newcounter{ionstage}
\renewcommand{\ion}[2]{\setcounter{ionstage}{#2}% 
  \ensuremath{\mathrm{#1\,\scriptstyle\Roman{ionstage}}}}
  
\newcommand\hii{\ion{H}{2}}
\newcommand\pos{\ensuremath{_{\mathrm{pos}}}}
\newcommand\los{\ensuremath{_{\mathrm{los}}}}

\newcommand\halpha{H${\alpha}$}
\newcommand\n{[\ion{N}{II}]$\lambda$6584}
\newcommand\oi{[\ion{O}{III}]$\lambda$5007}
\newcommand\s{[\ion{S}{II}]$\lambda$6737}
\newcommand\si{$\sigma$}
\newcommand\kms{$^{-1}$}


%%%%%%%%%%%%%%%%%%%%%%%%%%%%%%%%%%%%%%%%%%%%%%%%%%

%%%%%%%%%%%%%%%%%%% TITLE PAGE %%%%%%%%%%%%%%%%%%%

% Title of the paper, and the short title which is used in the headers.
% Keep the title short and informative.
\title[Turbulence in H II regions]{Turbulence in compact to giant HII regions}

% The list of authors, and the short list which is used in the headers.
% If you need two or more lines of authors, add an extra line using \newauthor
\author[J. García Vázquez et al.]{
J. García Vázquez,$^{1}$\thanks{E-mail: jgarciav1600@alumno.ipn.mx}
J. Zsargo,$^{1}$
W. J. Henney$^{2}$
and H. O. Castañeda$^{1}$
\\
% List of institutions
$^{1}$Escuela Superior de Física y Matemáticas, Instituto Politécnico Nacional, Ciudad de México, México.\\
$^{2}$Instituto de Radioastronomía y Astrofísica, Universidad Nacional Autonoma de México, Apartado postal 3-72, 58090 Morelia, Michoacán, Mexico\\
}

% These dates will be filled out by the publisher
\date{Accepted XXX. Received YYY; in original form ZZZ}

% Enter the current year, for the copyright statements etc.
\pubyear{2021}

% Don't change these lines
\begin{document}
\label{firstpage}
\pagerange{\pageref{firstpage}--\pageref{lastpage}}
\maketitle

% Abstract of the paper
\begin{abstract}
  Fluctuations of centroid velocities on the plane of the sky are a powerful tool for studying the turbulent dynamics of emission line regions.
  To characterize these fluctuations we apply a statistical analysis using the second-order structure function to archival \halpha\ observations of a diverse sample of 9 \hii{} regions.
  This regions are located in the Milky Way and other Local Group galaxies, and
  span more than two orders of magnitude in size and luminosity.
  By fitting a simple functional form of the structure function to our results
  we extract three parameters related to the fluctuations in each region: the
  total velocity dispersion \(\sigma\), which spans from 3 to 15 km s\(^{-1}\), the
  autocorrelation length \(r_0\), which covers 0.06 - 8 pc, and the power-law slope \(m\), which ranges from 0.82 to 1.95.
  The velocity dispersion is found to correlate primarily with luminosity,
  while the autocorrelation length correlates primarily with the size of the region.
  The power-law slope shows an apparent correlation with distance,
  but this is probably an artifact of poor spatial resolution for the most distant regions.
  A comparison between \(\sigma\) and \(\sigma_{POS}\) shows that this last value is at roughly the double for all regions.
  \textbf{Add a sentence about the conclusions!}
  % We study the turbulent velocity fluctuations in the ionized gas of four giant HII regions using the radial velocity field in the H$\alpha$ emission line. The data was acquired by long-slit spectroscopy and Fabry-Perot observations obtained with the William Herschel Telescope. Each velocity sample is statistically analyzed with the second order structure function. The correlations in our results shows evidence that the velocity field can be consider in a turbulent state. This turbulent motions are described by a power-law with a mean index of 1.7 for different sized regions at different distances. The physical interpretation of the power-law behavior consider that an energy cascade is taking place in the photoionized gas.
\end{abstract}

% Select between one and six entries from the list of approved keywords.
% Don't make up new ones.
\begin{keywords}
HII regions -- ISM
\end{keywords}


\definecolor{WillCommentColor}{rgb}{0.6,0.11,0.4}
\newcommand\WILL[1]{\textbf{\color{WillCommentColor}#1}}
%%%%%%%%%%%%%%%%%%%%%%%%%%%%%%%%%%%%%%%%%%%%%%%%%%

%%%%%%%%%%%%%%%%% BODY OF PAPER %%%%%%%%%%%%%%%%%%

\section{Introduction}
\WILL{I think it would be best to lead with a description of the dynamics of H II regions before mentioning the turbulence}.



It is known that the speed of sound in a HII region is of the order of \(c_{HII} \sim\) 13 km s\kms.
\citet{smith1970} discovered that the probable velocities of internal motions of extragalactic HII regions, \(\sigma_{W}\) (line width integrated over the whole HII region), were in the range of $\sim$ 19-34 km s\kms\ covering supersonic values.
\citet{skillman1984kinematics} found broad profiles in giant extragalactic HII regions (GEHRs) that have a FWHM\(>\)30km s\kms. 
This property is mainly used to distinguish between 'normal' and 'giant' HII regions. Other differences are their size, luminosity and their stellar population. 
For the velocity to be supersonic it is necessary a continuous mechanism to maintain the observed motions, as supersonic motions are expected to dissipate energy via shocks \citep{1994Ap&SS.216..285C}.
Also, this non-thermal broadening on the spectral emission lines has been suggested as evidence of disorder motions along the line-of-sight (LOS) of the ionized gas. 
Turbulence is suggested as a mechanism that can stir the medium and maintain the supersonic velocities. 
The large lengths of astrophysical objects, ranging in scales from 10$^{6}$ to 10$^{17}$ m \citep{2010ApJ...710..853C}, and the high Reynolds numbers (Re), in the order of 10$^{4}$ - 10$^{9}$, are the main properties that supports the idea of a turbulent interstellar medium.
Previous studies \citep{1999intu.conf.....F,2004ARA&A..42..211E,scalo2004interstellar} revealed that different phases of the interstellar medium show fluctuations on their velocity field that can be interpreted as evidence of random motions.

The usual picture considers that interstellar turbulence is driven on different scales by various energetic processes.
Also, it is thought that the formation of structures like filaments and shells in the diffuse medium is determined by large-scale turbulence.
Stars are the mainly mechanism proposed to account for the large input of kinetic energy through UV radiation, but the dominant energetic sources and their relation to turbulence still is unclear.

\citet{melnick1977,terlevich1981} were the first to shown an empirical relationship between the line width and the integrated properties of the HII regions as the diameter or the luminosity ($r \sim \sigma ^{\sim 2}$ ; $L_{H} \sim \sigma ^{\sim 4}$).
They conclude that GEHRs follow this relations as self-gravity systems like globular clusters or elliptical galaxies.
The work of \citet{1988A&A...201..199A} confirmed these relationships considering the discrepancies of previous investigations obtaining \(L_{H_{\alpha}} \propto \sigma^{3.9}\) and \(r \propto \sigma^{1.84}\).
\citet{2012MNRAS.422.3339W} have found that clumps and HII regions follow scaling relations over the range of z = 0-2 for \halpha\ size, velocity dispersion, luminosity and mass finding \(\sigma \propto r^{0.42}\), \(L_{H_{\alpha}} \propto r^{2.72}\) and \(L_{H_{\alpha}} \propto \sigma^{4.18}\). This results imply that the same process are seen at high redshifts and the current epoch in star forming regions. \citet{2015MNRAS.449.3568M} found that the relation is also found in dwarf galaxies with a \(L_{H_{\alpha}} \propto \sigma^{5}\).   

The idea of a virial system has been developed by \citet{1993ApJ...418..767T,munoz1996} into the \textit{Cometary Stirring Model} (CSM) with the aim to separate the mechanism that contribute to the line broadening.
The stellar winds also tend to be used for an explanation of the line broadening and turbulence in GEHRs \citep{1994ApJ...425..720C}.
This features seems to drive visually the regions with filaments, arcs and shells.
On their spectra double and multiple emission features are signature of these objects.
\citet{2020MNRAS.494...97S} argue that winds on ionizing clusters in GEHR must merge into a hot cluster because of the close proximity of the massive stars.
\citet{2019ApJ...871...17U} propose that stellar feedback is an important source of energy to maintain turbulence in nearby galaxies.

The classic turbulence theory \citep{kolm1} assumes a three-dimensional isotropic, incompressible and subsonic flow where an energy cascade is taking place.
The interstellar matter does not have those physical properties and our observations are in two-dimensions.
But as mentioned above the idea of an energy cascade is used to understand our observations.
The energy cascade goes from the largest vortices where the energy in injected until it reaches dissipation with the smallest vortices.
In the energy cascade three scales are defined based on their Reynolds number.
The energy injection scale (Re $\rightarrow \infty$), $L_{EI}$, the dissipation scale (Re $\rightarrow$ 1), $L_{D}$ and the inertial scale, $L_{I}$, between the previous two.
Since the energy distributions depends on the scale, the energy cascade concept is described with an energy spectrum, $E(k) \propto k$ where $k$ is the wave number for a determined scale $l$ ($k \equiv r^{-1}$).

The fluctuations associated with the centroid velocities on the plane-of-sky (POS) of photoionized regions have been characterized several times using the second order structure function (the Fourier transform of the energy spectrum, $E(k)$).
This studies span from galactic HII regions \citep{1986ApJ...300..624R} like Orion \citep{von1951methode,munch1958internal,castaneda1988,1992ApJ...387..229O,arthur2016turbulence} to giant HII regions \citep{1961MNRAS.122....1F,lagrois2009multi,lagrois2011}, like NGC 604 in the M 33 galaxy \citep{tanco1997,2019arXiv191203543M}.

The procedure using the structure function allow us to determine a distinction between homogeneous turbulence and pure random velocity fluctuations in the photoionized gas, but despite the previous evidence of this fluctuations not been random, little is understood about their nature.
Several problems arise in the study of these fluctuation.
Intrinsic with observations there is the projection of a three-dimensional function into the plane-of-sky \citep{von1951methode,munch1958internal}.
Other statistical techniques like velocity channel analysis (VCA) \citep{2000ApJ...537..720L} have been proved to better recreate the spectrum of the velocity field \citep{medina2014,arthur2016turbulence}.
More recently, the structure function was used with full 6D measurements (position-velocity) to study the motions of stars in the Orion Molecular Cloud Complex.
\citet{2021ApJ...907L..40H} argue that newborn stars should reflect the turbulent kinematics of their natal clouds.
Using this previous method they are free of the projected element that interfere with previous gas-based studies and their results agree with the scaling law proposed by \citet{1981MNRAS.194..809L}.

Despite different issues that are related with the nature of the problem of turbulence in HII regions, the vast majority of research using the second-order structure function consider individual regions and hence different methodology.
\citet{1987ApJ...317..686O} has determined the structure functions for a sample of galactic HII regions but isn't conclusive about them.
The individuality of previous works and the need to understand the relationship between the measured turbulent properties for different classes of HII regions has lead us to develop this investigation.

Using archival data we perform the second-order structure function on a sample of HII regions.
Our sample spans in size from 4 pc to 400 pc and has luminosities between 10\(^{37}\) - 10\(^{39}\) erg s\(^{-1}\); from galactic to extragalactic HII regions.
We apply an uniform methodology of a functional form of the structure function to characterize the turbulent using the  autocorrelation length, \(r_0\), and power-law slope, \(m\).
Our results are compared with physical properties of each region and are also compared with previous investigations.
With this work we aim to fill a gap with an uniform statistical analysis of turbulence between galactic and giant HII regions.

The paper follows the next order: in Section \ref{sec:HIIsample} we describe the observational data and some main kinematics properties of each region. In Section \ref{sec:met} we present the statistical method used in this work. Section \ref{sec:results} presents the structure function of each region along with the turbulent parameters obtained by a non-linear regression method. In Section \ref{sec:discussion} we discuss our findings in the context of previous works.   

\section{HII regions sample}\label{sec:HIIsample}

We use previous published observational data to measure the second-order velocity structure function for a number of nine HII regions. 
Table \ref{tab:Reg} shows the HII regions names
with the main physical properties of each region. 
These regions are located in the Milky Way, the Magellanic clouds and in the M 33 and NGC 6822 galaxies.

\subsection{Regions in the Milky Way and Magellanic Clouds}
\label{sec:regions-milky-way}

Orion is a galactic HII region located at a distance of 440 pc \citep{2008AJ....136.1566O}.
The main ionizing star is \(\theta^{1}\)Ori C (spectral type \(\sim\)O7) with fast stellar wind, and 3 B-type stars from the Trapezium cluster.
The position of Orion near the side of the Orion Molecular Cloud (OMC-1) make it an ongoing star formation region and a example of a blister HII region \citep{arthur2016turbulence}.
Its physical and kinematics properties along its stellar population are well documented in the review of \citet{2001ARA&A..39...99O}.

We use data from \citet{1987A&A...176..347H} and \citep{arthur2016turbulence} to study of turbulence in the Orion Nebula.
\cite{1987A&A...176..347H} observations were obtained using a 106 cm-Cassegrain telescope at Observatorium Hoher List. 
A Fabry-Pérot interferometer was used having the étalon a separation of 0.5 mm. 
Interferograms have been taken with different pointing directions of the telescope's optical axis in the nebula. 
The exposures are overlapped and fall into one square of a grid with a width of 1' centered at \(\theta^{1}\)Ori C.   

\citet{arthur2016turbulence} observations were initially published by \citet{2008RMxAA..44..181G}. The data were obtained with the echelle spectrograph attached to the 4 m telescope at Kitti Peak National Observatory. These observation cover 3 \(\times\) 5 arcmin\(^{2}\)of the Orion nebula and the data consist of 96, 300 arcsec North-South orientated slits at 2 arcsec intervals with a width of 0.8 arcsec. The velocity resolution is 8 km s\(^{-1}\).

The Carina nebula is located at a distance of 2.35$\pm$0.05 kpc \citep{2006ApJ...644.1151S}.
Young clusters like Trumpler 14 and 16, and Collinder 228 together form the Car OB1 association.
This OB association has more than 65 stars earlier than B0, and several Wolf-Rayet stars.
The most star is a luminous blue variable, $\eta$ Car \citep{Damiani:2016a}.
The stellar content has lead to identify the Carina nebula as a star formation region, and as such it presents several phenomena related with young stras as (\citet{2006ApJ...644.1151S} and reference therein): evaporating protoplanetary discs, erosion of large dust pillars and the triggering of a second generation of embedded stars, Herbig-Haro objects, to mention some. 
All of these phenomena are consequence of the UV radiation and stellar winds from star clusters in the HII region.

The Lagoon Nebula (M8, NGC6523) is located at a distance of 1 250 pc \citep{2005A&A...430..941P} and contains the young stellar cluster NGC6530.
The region is illuminated by different massive stars of spectral types O and B, the hottest being 9 Sgr, type O4V \citep{Damiani:2017b}.
Optically, the brightest part of the nebula is called 'Hourglass nebula' which surrounds and obscures the star Herschel 36, type O7 \citep{1986AJ.....91..870W}. 
The velocity field shows several expanding shells, related to the cluster NGC 6530, the O stars 9 Sgr and Herschel 36, and the massive protostar M8East-IR. A blueshifted, neutral layer is also in very good agreement with predictions of champagne-flow models of blister HII regions \citep{Damiani:2017b}.

Archival data from \citet{Damiani:2016a} and \citet{Damiani:2017b} are use to study the Carina nebula and Lagoon nebula, respectively. 
The ESO VLT/FLAMES multi-fiber spectograph was used for observations (Gaia-ESO Survey internal release \(iDR4\)) of Lagoon nebula  and Carina nebula \citep{2002Msngr.110....1P}. 
Mainly the spectra were obtained with the Giraffe HR15N setup (\(R \sim19 000\), wavelength range 6444-6818 A) which allow to observe \halpha\ emission line. 
The Survey targets are low-mass stars ranging down to magnitude V \(\sim\) 18.5.

Using observations from \citet{Castro:2018a} (private communication) we study the Tarantula Nebula (30 Doradus).
This nebula us located in the LMC (Large Magellanic Cloud) at a distance of 49.9 kpc \citep{2013Natur.495...76P} and is the most luminous star-forming complex in the Local Group \citep{1984ApJ...287..116K}.
It host the star forming region NGC 2070 that contains the massive star cluster R136.
NGC 2070 was observed as part of the MUSE SV (Science Verification) programme at the Very Large Telescope (VLT). 
Four overlapping fields, each with an individual field of 1' \(\times\) 1' and a pixel scale of 0.2''/pixel were observed. 
The data were reduced using the MUSE pipeline based on ESOREX recipes, with final astrometric calibration using the catalogue of \citet{1999A&A...341...98S}. 
The final spectra span 495-9366 A, with a resolving power of \(R \sim3 000\) around \halpha\  \citep{Castro:2018a}.

NGC 346 is the most active star-formation region in the SMC (Small Magellanic Cloud) located at a distance of 60 kpc. 
It has more than 30 O stars that ionize N66, the largest HII region \citep{2011ApJ...740...10D}.
\citet{2008ApJ...688.1050G} propose an expanding H II region or bubble blown by the winds of the massive progenitor as a mechanism that shapes the recent star formation in this region in addition to the photoionizing process of the OB stars. 
This mechanism is similar to shell-like H ii regions, with a central cluster in a cavity, and with ongoing star formation triggered around their periphery.

\subsection{Regions in M33 and NGC 6822} 

NGC 604 is the brightest giant HII region in the M33 galaxy visually dominated by a big loop, many shells and different size filaments.
It is located at the distance of 840 kpc (1'' = 4.1 pc) \citep{2015KamKinematics} with a length of 400 pc.
It has a luminosity in L$_{H_\alpha}$ = 10$^{39.42}$ erg s$^{-1}$ \citep{2002MNRAS.329..481B} and in L$_{X}$=9.3$\times$ 10$^{35}$ erg s$^{-1}$ \citep{2008ApJ...685..919T}.
\citet{2012ApJ...761....3M} derive an average age of 4$\pm$1 Myr and a total stellar mass of 1.6$^{+1.6}_{-1.0} \times$ 10$^{5}$M$_{\odot}$ for the region.

The stellar population has been identified and classified by \citet{1996ApJ...456..174H} (and references therein).
Recently, \citet{2011MNRAS.411..235E} focused on studying the Wolf-Rayet (WR) and red super giants (RSGs) stars population.
\citet{2012AJ....143...43F} focused on the massive young stellar objects (MYSOs).
 The average ages for the stars, was determined to be from 3 to 5 Myr \citep{1996ApJ...456..174H} and with the RSGs as an older population have an age of 12.4 $\pm$ 2.1 Myr \citep{2011MNRAS.411..235E}.
This suggest that two episodes of star formation, at least, have taken place in the GEHR.
\citep{1984A&A...141...49H} identified a gradient from the west to the SE of the nebula of 30 km s$^{-1}$.

NGC 595 is the second brightest giant HII region in the M33 galaxy.
The electronic temperature is 7,670 $\pm$116 K \citep{2010MNRAS.402.1635R}.
\citet{1983A&A...119..185V} made an estimate of the total masses of HI and HII in the nebula to obtain values of M$_{HI}=$ 1.2 $\times$ 10 $^{6}$ M$_{\odot}$ and M$_{HII} =$ 4.6 $\times$ 10 $^{5}$ M$_{\odot}$, and a ionizing luminosity of the star cluster is estimated at 5 $\times$ 10$^{50}$ erg s $^{1}$.
\citet{1996AJ....111.1128M} estimate an age of 4.5 Myrs consistent with \citet{1993AJ....105.1400D}.

It has approximately 250 stars of the OB type, 13 Supergiants and 9 WR \citep{1996AJ....111.1128M}, ten WRs confirmed with spectroscopy \citet{1993AJ....105.1400D}, nine of the WN type and a WC, located near the bright core of the region.

The dwarf irregular galaxy, NGC 6822, is located at a distance of 500 kpc (1'' = 2.4 pc) \citep{1996AJ....112.1928G} and the brightest HII regions are Hubble V (HV) and Hubble X (HX).
HV has a characteristic length of 100 pc and a luminosity at its core of in L$_{H_\alpha}$ = 1 $\times$ 10$^{49}$ H$\alpha$ photons s$^{-1}$ \citep{1999PASP..111.1382O}.
HX has a characteristic length of 140 pc and a luminosity at its core of in L$_{H_\alpha}$ = 2.4 $\times$ 10$^{49}$ H$\alpha$ photons s$^{-1}$ \citep{1999PASP..111.1382O}.

%Two recognized OB associations \citep{1991ApJ...379..621H,1992AJ....104.1374W} are identified within the region of HV and HX. HV is located near 'Hodge OB 8', and HX is located near 'Hodge OB13'  \citep{1999PASP..111.1382O}.

The archival data used in this work for the extragalactic giant regions was obtained with the Fabry Perot TAURUS-II instrument in the 4.2-m William Herschel Telescope (WHT) of del Roque de los Muchachos Observatory (ORM) in La Palma, Spain and was retrieved from the La Palma archive\footnote{\url{http://casu.ast.cam.ac.uk/casuadc/ingarch}}.
\citet{sabalisck1995supersonic} mentioned the main aspects of the observations (1990 July 31).
The two-dimensional imaging spectroscopy was captured using an IPCS-II detector and the 125 $\mu$m etalon.
The cumulative integration time per frame is 36 s with a seeing value of 1''.
The format of the data cube was of 256*256*100 with a spatial scale of 0.26 pixel$^{-1}$.
Finally, the reduction and analysis was done with the packages TAUCAL y TAUFITS \citep{1992ASPC...25..445L}.
The NGC 604 TAURUS-II observations were used by \citet{sabalisck1995supersonic}, \citet{tanco1997} and \citet{2019arXiv191203543M}.

%\citet{2002MNRAS.329..481B} 39.42 38.95 38.21 38.30  39.46 37.36  37.08

\begin{table*}
\begin{center}\caption{Sumary of properties \citep{1984ApJ...287..116K} \citep{1986ApJ...300..624R}}
\begin{tabular}{ccccccccc}\hline
HII    &  Dist.  & Diameter & log L(H$_{\alpha}$) & SFR      &  $ \langle N_{e} \rangle_{rms}$    & M(H$^{+}$) & Filling & \(\sigma_{W}\) \\
Region     &  [kpc] &  [pc]     &  [erg/s]            & [ ]      & [cm$^{-3}$] & [M$_{\odot}$] & Factor & [km/s] \\ \hline
NGC 604   &   840  & 400     &    39.65     & 0.011    & 3  & 7 $\times$ 10$^{5}$ & 0.1 & 23.1 \\
NGC 595   &   840  & 400     &    39.36     & 0.009    & 4  & 4.6 $\times$ 10$^{5}$ & & 27.1 \\
Hubble X  &   500  & 160     &    38.6      &  0.004   &  5 & 23 000& 0.02 & 13.4 \\
Hubble V  &   500  & 130     &    38.87     &  0.005   &  8 &    & & 14.7 \\
30 Dor    &   50   & 98.9    &    39.76     & 0.0130   & 250&    & & 31.7 \\
Carina    &   2.35 & 15      &    39.6      & 0.01     & 500&    & 0.05 & \\
NGC 346   &   61.7 & 64      &    38.67     & 0.0021   & 100&    &     & \\
Lagoon    &   1.25 & 25      &    37.47     & 0.0001   &  60& 900& 0.012 & \\
Orion     &   0.4  & 5       &    37        & 0.000053 & 150& 50 & 0.027 & \\\hline
\end{tabular}\label{tab:Reg}
\end{center}
\end{table*} 


%%%%%%%%%%%%%%%%%%%%%%%%%%%%%%%%%%%%%%%%%%%%%%%%%%%%%%%%%%%%%%%%%%%%%%%%%%%%%%%%%%%%%%%%%%%%%%%%%%%%
%%%%%%%%%%%%%%%%%%%%%%%%%%%%%%%%%%%%%%%%%%%%%%%%%%%%%%%%%%%%%%%%%%%%%%%%%%%%%%%%%%%%%%%%%%%%%%%%%%%%%%%

\section{Methods}\label{sec:met}

\subsection{The second order structure function}

One of the fundamental properties of turbulent clouds must be spatial fluctuations in velocity and density \citep{1984ApJ...277..556S}.
They are observed as projected spatial fluctuations in the average line-of-sight velocity and integrated column density.
By means of describing the statistical properties of these fluctuations they can be assumed as a product of a stochastic field statistical tools.

The statistical function we use to study the velocity field of the HII regions is the second-order structure function, $B(r)$, defined as:

\begin{equation}\label{eq:S}
B(\boldsymbol{r})=\dfrac{\sum[V_{r}(\boldsymbol{x}+\boldsymbol{r})-V_{r}(\boldsymbol{x}) ]^{2}}{N(\boldsymbol{r})}
\end{equation}

where, $V_{r}(\boldsymbol{x})= V_{obs}(\boldsymbol{x})-\langle V_{obs}(\boldsymbol{x}) \rangle$ and N($\boldsymbol{r}$) is the number of points at each separation \(r\), called lag.

As a justification for the use of the structure function we can consider that any relation between the line width and line-of-sight depth can be interpreted in terms of a correlation between velocity spread and scale size.
This follows if we have an optically thin spectral line where it will have a width which is related to the total radial velocity dispersion of all fluctuations sampled along the line-of-sight \citep{1984ApJ...277..556S}. 
In these sense the structure function is used as a tool to investigate the projection of a three.dimensional correlation function onto the plane of sky \citep{arthur2016turbulence}.

We also use the functional form of the structure function:

\begin{equation}\label{eq:b}
b(r)=2[1-C(r)]\zeta+\eta
\end{equation}

where $C(r)=1/[1+(r/r_{0})^{m}]$ \citep{1984ApJ...277..556S,arthur2016turbulence}.
The term $r_{0}$ is the correlation length of the turbulence, the value of \(r\) in pc where equation \ref{eq:S} reach the value of \(\sigma^2\), and $m$ is the index that adjust our results.
The term $\zeta$ is a correction for small scales considering a Gaussian seeing, $s_{0}$, and has the form $tanh^{2}[(r/2s_{0})^2]$, and \(\eta\) is the observational noise. 
In this idealized case, at scales larger than r$_{0}$ the structure function flattens as it tends towards the asymptotic value of 2$\sigma$ \citep{arthur2016turbulence}.

\subsection{Application to our HII regions sample}\label{sec:apply}

The results and the errors of $r_{0}$ and $m$ are obtained using equations \ref{eq:S} and \ref{eq:b} in a chi-square (\(\chi^2)\) statistic in order to do a non-linear least-squares fit of a model to our data \citep{newville_matthew_2014_11813}. 
In this way we avoid one fundamental problem in the interpretation of the structure function results. The slope in the majority of previous studies and hence, the correlation length, is obtained by eye. 
The traditional way to find it is to determine a particular range of ascending values in the structure function results. 
If this increase in values is stopped it is commonly assumed that one particular range has been found. 
In some cases this procedure keeps going up to large scales, giving two or more inertial regimes;this has lead to the interpretation of multiple energy injection scales.
Since the previous described method isn't conclusive we choose the chi-square (\(\chi^2)\) statistic to present the significance of our results. 
This way we can maintain consistency in our analysis.


\section{Results}\label{sec:results}

\subsection{Data}

The Figure \ref{fig:hist} shows the histograms of each individual sample with the variance  \(\sigma^{2}\), along the number of points of each sample.

%\begin{table*}
%  \begin{center}
%    \caption{General statistical properties of the centroid velocities.} 
%\begin{tabular}{cccc}\hline
%HII         &$\mu$   &$\sigma^{2}$       &No. of  \\
%Region      &[km/s]  &[km$^{2}$/s$^{2}$] &points  \\ \hline
%NGC 604     &3.56    &54.63              &9 510    \\
%NGC 595     &2.82    &44.13              &8 765    \\
%Hubble V    &-1.23   &7.87               &5 832   \\
%Hubble X    &-5.95   &12.9               &6 925   \\
%30Dor       &265.56* &252.79             &363 059 \\
%Carina      &-8.83   &17.89              &855     \\
%NGC 346     &        &31.36              &112 796 \\
%Lagoon      &-6.26   &7.53               &1 176   \\
%Orion Large &-1.25   &10.43              &359     \\
%Orion Small &        &9.61               &        \\ \hline
%\end{tabular}\label{tab:Data}
%\end{center}
%\end{table*} 

\begin{figure*}
\centering
\begin{multicols}{3}
\includegraphics[width=1.5in]{Figures/Hist/604}\par
\includegraphics[width=1.5in]{Figures/Hist/595}\par
\includegraphics[width=1.5in]{Figures/Hist/346}\par
\end{multicols}
\begin{multicols}{3}
\includegraphics[width=1.5in]{Figures/Hist/Hubble-V}\par
\includegraphics[width=1.5in]{Figures/Hist/Hubble-X}\par
\includegraphics[width=1.5in]{Figures/Hist/30Dor}\par
\end{multicols}
\begin{multicols}{3}
\includegraphics[width=1.5in]{Figures/Hist/M8}\par
\includegraphics[width=1.5in]{Figures/Hist/Car}\par
\includegraphics[width=1.5in]{Figures/Hist/OrionS}\par
\end{multicols}
\caption{Histograms of the centroid velocities of our sample of HII regions. }
\label{fig:hist}
\end{figure*}

\subsection{HII regions structure functions}


Table \ref{tab:Res} presents the three parameters we obtain from the analysis on the fluctuations of the velocities using the second order structure function.
The second column is the velocity dispersion \(\sigma\) of each sample of centroid velocities.
The third column is the correlation length \(r_0\).
The index \(m\) is shown in the forth column;
this index is used as a reference for comparison between theory and observations.
In the fifth column we show the reduced chi-square value obtained using a non-linear regression algorithm. 
Finally we show the mean velocity dispersion across the line-of-sight \(\sigma_{LOS}\) with the aim to investigate how statistically homogeneous the regions are.

%In the appendix \ref{sec:StructFunct} the Figures \ref{fig:SF604}-\ref{fig:SFM42} show the second-order structure functions of the different HII regions for the \halpha\ emission line.

The power-law fit in the structure function for the both regions in the M33 galaxy in is performed above scales of 1.7 pc.
For NGC 604 the index is correspond to 1.95 with a correlation length of 8.75 pc. For NGC 595 the values obtained are 1.75 and 9 pc.
For NGC 6822 regions, Hubble X and Hubble V, the $m$ index is 1.7 and 1.62, with a correlation length of 3.62 and 2.7 pc, respectively.
The 30Dor nebula \citep{Castro:2018a} has a correlation length of 2.7 pc and an index of 1.22.
NGC 346 correlation length is 1.4 pc and its index is 1.13.

The Carina observations allow us to measure a blue shifted and a red shifted layer \citet{Damiani:2016a}. It is possible to combine the moments from the sum of two components \citep{2008RMxAA..44..181G}, we use this combined sample for the structure function analysis.
Carina has an index of 0.82 with a correlation length of 0.94 pc.
The galactic region Lagoon nebula \citep{Damiani:2017b} presents a correlation length of 1.78 pc and a index of 1.56.

For Orion we have the opportunity to study two different observations. \citet{1987A&A...176..347H} observations allow us to study the large scale of the nebula, covering 4 pc, while \cite{arthur2016turbulence} allow us to study the small scales up to 0.4 pc. Large scales results shows an index of 0.9 with a correlation length of 0.84 pc while the small scales have 1.44 and 0.062 pc.
  


The highest \(\sigma\) is 15.8 km s\(^{-1}\) for 30Dor.
This is two times higher than NGC 604 that has a value of 7.4 km s\(^{-1}\), followed by NGC 595 with 6.64 km s\(^{-1}\).
NGC 346 is next with a value of 5.6 km s\(^{-1}\), followed by Carina with 4.22 km s\(^{-1}\).
Hubble X and M42 have a value of 3.59 and 3.23 km s\(^{-1}\), respectively.
Hubble V has 2.8 km s\(^{-1}\) and last mof all sample is Lagoon with 2.7 km s\(^{-1}\).

\begin{figure*}
\centering
\begin{multicols}{3}
\includegraphics[width=2in]{Figures/SFplots/604.pdf}\par
\includegraphics[width=2in]{Figures/SFplots/595.pdf}\par
\includegraphics[width=2in]{Figures/SFplots/346.pdf}\par
\end{multicols}
\begin{multicols}{3}
\includegraphics[width=2in]{Figures/SFplots/Hubble V.pdf}\par
\includegraphics[width=2in]{Figures/SFplots/Hubble X.pdf}\par
\includegraphics[width=2in]{Figures/SFplots/tarantula.pdf}\par
\end{multicols}
\begin{multicols}{3}
\includegraphics[width=2in]{Figures/SFplots/Carina.pdf}\par
\includegraphics[width=2in]{Figures/SFplots/M8.pdf}\par
\includegraphics[width=2in]{Figures/SFplots/OS.pdf}\par
\end{multicols}
\caption{Second-order structure functions for the velocity centroid images of the \(H\alpha\) emission line investigated in this work. The first two rows consider extragalactic regions while the last row shows galactic regions.}
\label{fig:SFs}
\end{figure*}

\begin{table*}
\begin{center}\caption{Main results.
 a) \citet{tanco1997}
      b) \citet{2019arXiv191203543M}
      c) \citet{Castro:2018a}
      d) \citet{Damiani:2016a}
      e) \citet{Damiani:2017b}
      f) \citet{1987A&A...176..347H}
      g) \citet{arthur2016turbulence}.}
\begin{tabular}{ccccccccc}\hline
HII         &\(\sigma\) &\(r_0\)                     &\(m\)    &\(\langle \sigma_{LOS} \rangle \) & Previously\\
Region      &[km/s]     &[pc]                        &                           &  [km/s]& used in:\\ \hline
NGC 604     &7.39       &8.75\(^{+0.68}_{-0.75}\)    &1.95\(^{+0.12}_{-0.14}\)   &16.2  &a,b \\
NGC 595     &6.64       &8.99\(^{+0.58}_{-0.71}\)    &1.75\(^{+0.10}_{-0.10}\)   &18.3  &- \\
Hubble V    &2.80       &2.7\(^{+0.24}_{-0.14}\)     &1.62\(^{+0.11}_{-0.09}\)   &13.4  &- \\ 
Hubble X    &3.59       &3.62\(^{+0.18}_{-0.15}\)    &1.7\(^{+0.08}_{-0.09}\)    &12.3  &- \\   
30 Dor      &15.8       &2.69\(^{+0.08}_{-0.03}\)    &1.22\(^{+0.022}_{-0.007}\) &31.7* &c\\
Carina      &4.22       &0.53\(^{+0.12}_{-0.11}\)    &1.19\(^{+0.20}_{ -0.16}\)  &22.4  &d\\
NGC 346     & 5.6       &1.4\(^{+0.05}_{-0.75}\)     &1.13\(^{+0.03}_{-0.03}\)   &10.2* &-\\
Lagoon      &2.7        &1.78\(^{+0.07}_{-0.07}\)    &1.56\(^{+0.05}_{-0.05}\)   &13.6  &e\\ 
Orion Large &3.23       &0.84\(^{+0.37}_{-0.24}\)    &0.9\(^{+0.04}_{-0.04}\)    &6     &f \\
Orion Small &3.1        & 0.062\(^{+0.003}_{-0.001}\)&1.44\(^{+0.11}_{-0.01}\)   &6     &g \\\hline	  
\end{tabular}\label{tab:Res}
\end{center}
\end{table*}  

\subsection{Structure function power indices \(m\)}

It is possible that the index \(m\) is biased to \(m>\)1.6 for the extragalactic regions and \(m<\)1.5 for nearby regions.
This could be attributed to a data quality issues because the inability to resolve distant regions in the same way we do with nearby objects.
Since we are not observing all the inertial range on distant regions the \(m\) parameter  should be taken with some caution.

We can compute the $m_{3D}$ index from the two dimensional \(m\), if we assume that the $m_{3D}$ index lies between the limits of projection smoothing and a sheet-like distribution. 
For our observations we have 0.82 $<$ m$_{3D}$ $<$ 1.95.
This is done following the procedure by \cite{arthur2016turbulence} assuming the relationship m$_{2D}$ - 1 $<$ m$_{3D}$ $<$ m$_{2D}$.
This give us a considerable large range of 1. 
Though \cite{arthur2016turbulence} are comparing emission lines and not regions we can conclude that the index \(m\) still is far away as a unique parameter for measuring turbulence and that the three-dimensional behavior would probably encompass a wide range of indices as seen in the two-dimensional observations.


\subsection{Significance on the length scale \(r_0\)}

Ideally the correlation length \(r_0\), should represent the scale of dominant energy injection. 
Since it is unclear what are the mechanisms driving the turbulence, thought a lot of options are candidates, it is difficult to pinpoint one particular as main responsible. 
The possibility that the correlation length reveals the answer to the cause of observed motions is a long shot, but it is clear that each velocity field has a corresponding \(r_0\) value. 
The range covered in our results goes from 0.06 to 9 pc, a small range in comparison with the two order magnitude difference in size and luminosity.

\subsection{Correlations between turbulent parameters}

\begin{figure}
\centering 
\includegraphics[width=2in]{Figures/mr0.pdf}
\caption{ }
\label{fig:mr0}
\end{figure}


%%%%%%%%%%%%%%%%%%%%%%%%%%%%%%%%%%%%%%%%%%%%%%%%%%%%%%%%%%%%%%%%%%%%%%%%%%%%%%%%%%%%%%

\section{Discussion}\label{sec:discussion}

\subsection{Comparison with previous structure functions}

In Section \ref{sec:apply} we mentioned the reduced chi-square method we are using for determining the correlation length and the index of the slope. Since this is the first time this method is applied to a sample of HII regions (at least for our knowledge), the results are bounded to be different from previous investigations, including the one that have the same observations as \citet{arthur2016turbulence} and \citet{2019arXiv191203543M}. We will focus on the interpretation of each investigation results and leave the comparison of values as second issue.

A complete analysis on different emissions lines of the structure function on Orion was carried out by \citet{arthur2016turbulence} and reference therein. 
They observed a tendency that higher ionization lines presents higher values on the magnitude dispersion and the steepness of the structure function slope. 
This results just consider one component on the Gaussian fit. 
The power-law index obtained by \citet{arthur2016turbulence} are \(m \sim 1.2 \pm 0.1\) for \halpha\ and [OIII] emission lines and for [SII] they found \(m \sim 0.8 \pm 0.1\). 
The correlation length for all lines is \(\approx\) 0.05 pc. For the same \halpha\ observation we obtained a value of 1.44 and a correlation length of 0.062 pc. 
As a comparison for the large scales \citep{1987A&A...176..347H} the value is 0.9 with a considerable agreement with previous results. 
The change in the index could be attributed to the multiple scales of energy injection that are present in the nebula.

The first attempt at characterizing turbulence in a giant extragalactic HII region, GEHR, was done by \citet{1961MNRAS.122....1F} with the 30 Doradus complex in the LMC, finding no indication of turbulent motions between scale of 10 and 100 pc.
\citet{2019arXiv191203543M} also investigated the structure function in 30 Doradus finding on their results that the structure function is entirely flat on scales from 3 to 200 pc.
In the context of the high-resolution observations it is possible to see why their results do not reflect the turbulent structure of the nebula; their observations cover a region where the large-scale velocity fluctuations are uncorrelated.
There is a good agreement with the single-Gaussian fit results from \citet{2019arXiv191203543M} (Figure 15 Top on their work), while \citet{1961MNRAS.122....1F} results do not overlap with ours.
From the previous this is the first time turbulence structure is found in 30 Doradus through the structure function.

There is broad agreement in our results and previous studies considering NGC 604 from \citet{tanco1997} and \citet{2019arXiv191203543M}, and using the same TAURUS-II data.
Despite this general agreement the interpretations in each study is different.
\citet{tanco1997} conclude about a double regime acting on the kinetic energy spectrum.
According to this double cascading, turbulence is being forced at scales of \(\approx\)10 pc while and energy cascade has developed down to the smallest scales and other, as an inverse cascade, extends up to scale of \(\approx\)70 pc.
Interesting enough, \citet{tanco1997} mentioned various characteristic scale lengths, the most notorious is the \(\approx\)10 pc, a value close to our correlation length of \(\approx\)9 pc, and they interpret it as possible source on energy associated with the expansion of shocks coming from wind bubbles.  
None of the previous studies present a power-law index for the structure function to compare our results.
\citet{2019arXiv191203543M} addresses some issues with TAURUS II data while comparing profiles between the previous instrument TAURUS I where there exist a difference between observations (see section 4.1 on their paper).
As we agree with them in the importance of observing NGC 604 at higher spatial resolution, we have prove that a consistent method in analyzing structure function is key to obtaining trustworthy results.

For our NGC 595 \halpha\ emission line results there is no correspondence in the structure function between our results and previous investigations.
Our observations cover the brightest part of the region and have smaller resolution.
\citet{lagrois2009multi} and \citet{lagrois2011} structure functions results does not cover scales $<$10 pc, where in our results the correlation is taking place.
\citet{lagrois2011} used the data from \citet{lagrois2009multi} to complement the structure function analysis with the auto correlation function and filters to the velocity field to get rid of non-turbulent movements.
Their sigma is 5.92$\pm$0.15 km/s with a correlation length is 43 pc.
The high difference is because the consider that this happen at 2$\sigma$ (equation 13 on their paper).
The power-law index they provide is 1.55$\pm$0.01.

\subsection{Correlations between physical properties and turbulent parameters}

As we mentioned in the introduction, the scaling relations between parameters like size and luminosity are common tools to study HII regions along dwarf galaxies and high-redshift star formation regions; in the past those relationships, particularly in GEHRs, made them candidates as standard candles. 
We apply these methods with the turbulent parameters \(r_0\) and index \(m\).

Using the first three columns in Tables \ref{tab:Reg} and \ref{tab:Res}, we obtain the Pearson correlation coefficient $r$. Between luminosity (L) and \(\sigma\) it is 0.71, between distance (D) and \(m\) the value is 0.73, and finally between size (R) and \(r_0\) we have 0.96. 
By means of a significance test and assuming that a uncorrelated sample produces our results, for the first two relationships we have a probability value $p$ of 0.02 and for the remaining one of <10\(^{-5}\).
Applying and ordinary least-square regression (OLS) we find a form Y = aX + b; this values along the previous ones is shown in Table \ref{tab:RestStats}.

The relationship between L vs \(\sigma\) is one of the most studied. 
This relationship help us understanding how the star formation affects the gas motions and how star formation is regulated by feedback processes. 
Figure \ref{fig:sigvsl} shows different studies of the L vs \(\sigma\) relationship \citep{1988A&A...201..199A,Rozas:2006b,2015MNRAS.449.3568M}. 
In this case it is normally used the deconvolution of the observed lines taking account thermal broadening and instrumental broadening leaving the \(\sigma_{LOS}\). 
For the data present in Figure \ref{fig:sigvsl}, excluding \(\sigma_{POS}\), we obtain using the same OLS \(log \sigma_{LOS} = 0.145 \pm 0.008 (logL_{H\alpha}) - (4.42 \pm 0.336) \).
Since the R-squared is 0.596 it is not possible just to invert the OLS results for a comparison with the more classical relationship. Inverting axes we obtain \(log L_{H\alpha} = 4.09 \pm 0.23 log (\sigma_{LOS}) + (34.17 \pm 0.32) \), being in accord to previous relationships. 

It is also possible to use our results from Table \ref{tab:Res} for a comparison with other relationship, in this case considering the separation \(r\).
Using the mean slope index \(\langle m \rangle \), 1.4, we can assume \(\Delta V^{2} \propto r^{1.4} \) and obtain \(r \propto V^{1.5} \), close to what previous studies have determined. 

\begin{table*}
\begin{center}\caption{Parameters of the linear regression (Y = aX + b).}
\begin{tabular}{cccccc}\hline
Y                    &X         &a               &b                &r     & p                \\ \hline
log \(r_0\)          &log \(R\) &0.65\(\pm\)0.06 &-0.83\(\pm\)0.11 & 0.96 & <10\(^{-5}\)    \\ 
log \(\sigma\)       &log \(L\) &0.16\(\pm\)0.05 &-5.54\(\pm\)2.13 & 0.71 & 0.02             \\
log \(m\)            &log \(D\) &0.05\(\pm\)0.02 &-0.05\(\pm\)0.03 & 0.73 & 0.02  \\ 
*log \(\sigma_{LOS}\) &log \(L\) &0.145\(\pm\)0.008 &-4.42\(\pm\)0.33 & 0.77 & <10\(^{-5}\)  \\ 
*log \(L\) &log \(\sigma_{LOS}\) &4.09\(\pm\)0.23 &34.17\(\pm\)0.32 & 0.77 & <10\(^{-5}\)  \\ 
log \(\sigma\)       &log \(\sigma_{LOS}\) &0.72\(\pm\)0.27 &-0.14\(\pm\)0.31 &0.68 & 0.02 \\ \hline
\end{tabular}\label{tab:RestStats}
\end{center}
\end{table*} 

\begin{figure}
\centering 
\includegraphics[width=3in]{Figures/lvss.pdf}
\caption{}
\label{fig:sigvsl}
\end{figure}

\subsection{Relationship between plane-of-sky and line-of-sight velocity dispersion.}

The analysis employed until now is not conclusive about the true velocity fluctuations observed in each nebula. 
As pointed out by \citet{arthur2016turbulence} other techniques also allow us to study these fluctuations. 
Following their procedure, it is possible to see from Table \ref{tab:Res} that the line-of-sight velocity dispersion is roughly twice the plane-of-sky velocity dispersion.
In the case of homogeneous turbulent velocity we can consider the effect of 'project smearing', \citet{arthur2016turbulence} and reference therein, where the projection from three to two dimensions over a line-of-sight depth \(H\) reduces the plane-of-sky amplitude if \(r_{0} < H\).
With a value of \(\sigma / \sigma_{LOS}\) = 0.5 we obtain \(r_{0} / H \approx\) 0.02-0.1 \citep{1984ApJ...277..556S}. 

%Trying to answer if the observed turbulence can account for the non-thermal line broadening can be used as a diagnostic...


\begin{figure}
\centering 
\includegraphics[width=2in]{Figures/sigmas.pdf}
\caption{Log \(\sigma_{LOS}\) vs Log \(\sigma\) plot. The values are taken from Table \ref{tab:Res}. The solid-gray line indicate perfect correlation and the dotted-black line indicate the relation in Table \ref{tab:RestStats}. The size of the markers indicate the distance to the region, the bigger the marker the farther the region. }
\label{fig:sigvssig}
\end{figure}

\subsection{What turbulence scales can tell us about kinematics of the ionized gas?}

In the Table \ref{tab:ResII} the second column show the separation where the value 2\(\sigma\) is reached.
We also show the ratio \(r_{max}/r_{0}\), where \(r_{max}\) is the maximum separation or lag in pc; this value is related to the size of each sample box. 
The forth column shows the Taylor scale, \(l_T\), which measures the average spatial extent of velocity gradients. \citet{1999ApJ...524..895M} define this scale as:

\begin{equation}\label{eq:TS}
l_T=\dfrac{\sigma r_s}{\sigma_{r_s}}
\end{equation}

where \(r_s\) is the smallest separation \(r\) and \(\sigma_{r_s}\) the value of the structure function at that separation. 

It is possible to calculate how this Taylor micro-scale is related to the correlation length by:

\begin{equation}\label{eq:TS1}
l_T=\dfrac{r_s}{\sqrt{2}}(\dfrac{r_0}{r_s})^{m/2}
\end{equation}  

Finally in the table we present the Reynolds Number...

\begin{table}
\begin{center}\caption{Other significant turbulent scales}
\begin{tabular}{cccccc}\hline
HII         &r$_{2\sigma}$ &\(r_{max}\)/\(r_0\)  &\(l_T\) & \(Re\)* \\
Region      &[pc]          &                     &[pc]    &   \\ \hline
NGC 604     &28.73         &15.49                &5.35    &   \\
NGC 595     &18.50         &21.01                &5.57    &   \\
Hubble V    &11.61         &19.95                &1.80    &   \\ 
Hubble X    &21.99         &21.67                &1.97    &   \\   
30 Dor      &11.26         &8.99                 &0.59    &   \\
Carina      &1.93          &31.55                &0.402   &   \\
NGC 346     &5.78          &10.73                &0.39    &   \\
Lagoon      &6.14          &10.68                &0.66    &   \\ 
Orion Large &-             &2.98                 &0.24    &   \\
Orion Small &0.162         &10.42                &0.019   &   \\ \hline	  
\end{tabular}\label{tab:ResII}
\end{center}
\end{table}  

With the aim to relate each \(r_0\) with each region we invoke previous studies on the kinematics of the ionized gas. 
\citet{sabalisck1995supersonic} studied NGC 604 revealing that the kinematics of the region behave different considering different scales.
High emission knots on the region encompasses supersonic global velocity dispersion values of 14 $<$ \(\sigma_{LOS}\) $<$ 20 km s\(^{-1}\).
This knots are well fitted with Gaussian profiles thus, defining the kinematic core \citep{munoz1996}.
A characteristic radius for them is \(\sim\) 7 pc.
\citet{yang1996} identified 5 expanding shells within NGC 604, and one of them (Shell 3 on their paper) fall into our observations.
This shells is 40 pc in diameter with an expansion velocity above 50 km s\(^{-1}\).
This scales are related to the presence of strong stellar winds caused by massive stars.
For the NGC 604 case, there is no clear relation between the \(r_0\) and some particular energy injection mechanism...
%it is also the nearest analog of more extreme star-forming regions, such as 30 Doradus in the Large Magellanic Cloud.

%%%%%%%%%%%%%%%%%%%%%%%%%%%%%%%%%%%%%%%%%%%%%%%%%%%%%%%%%%%%%%%%%%%%%%%%%%%%%%%%%%%%%%%%%%%%%%%%%%%%%%%%%%%%%%%%%%%%%%


\section{Conclusions}\label{sec:conclusions}

%\begin{enumerate}
%    \item The giant HII regions in  M 33, NGC 604 and NGC 595 shows an index of 1.7 and 1.55, respectively. 
    
%    \item The structure functions are presented for the first time for Hubble X and Hubble V, in NGC 6822. Hubble V shows and index of 1.65 and a correlation length of 3 pc, while Hubble X results are 1.6 and 4 pc.
    
%    \item The structure function of 30 Dor is presented for the first time using high-resolution observations that allows to observe scales $>$0.1 pc. A correlated structure function is obtained with a correlation lenght of 2.7 pc and a power-law index of 1.15. This new results show that previous studies on 30 Dor structure functions were analyzing large scales where no velocity fluctuations are present.

    
%\end{enumerate}

%%%%%%%%%%%%%%%%%%%%%%%%%%%%%%%%%%%%%%%%%%%%%%%%%%%%%%%%%%%%%%%%%%%%%%%%%%%%%%%%%%%%%%%%%%%%%%%%%%%%%%

\section*{Acknowledgements}

JGV acknowledges and thanks CONACYT for the PhD research scholarship. We are grateful to Norberto Castro Rodríguez for providing maps of emission line velocity moments for 30 Doradus derived from MUSE-VLT observations.

%%%%%%%%%%%%%%%%%%%%%%%%%%%%%%%%%%%%%%%%%%%%%%%%%%
%%%%%%%%%%%%%%%%%%%% REFERENCES %%%%%%%%%%%%%%%%%%

\bibliographystyle{mnras}
\bibliography{bibphd}

%\clearpage

%%%%%%%%%%%%%%%%%%%%%%%%%%%%%%%%%%%%%%%%%%%%%%%%%%
%%%%%%%%%%%%%%%%% APPENDICES %%%%%%%%%%%%%%%%%%%%%
%%%%%%%%%%%%%%%%%%%%%%%%%%%%%%%%%%%%%%%%%%%%%%%%%%

%\appendix

%\section{Velocity maps}\label{sec:VelMaps}

%The Figures \ref{fig:M604}-\ref{fig:MOrion} show the two-dimensional radial velocity map of each region.

%\begin{figure}
%\centering 
%\includegraphics[width=2.5in]{Figures/M6T}
%\caption{Radial velocity maps of NGC 604 in H$\alpha$.  }
%\label{fig:M604}
%\end{figure}

%\begin{figure}
%\centering 
%\includegraphics[width=2.5in]{Figures/M5T}
%\caption{Radial velocity map of NGC 595 in H$\alpha$.}
%\label{fig:M595}
%\end{figure}

%\begin{figure}
%\centering 
%\includegraphics[width=2.5in]{Figures/MV}
%\caption{Radial velocity map of Hubble V in H$\alpha$.}
%\label{fig:MHV}
%\end{figure}

%\begin{figure}
%\centering 
%\includegraphics[width=2.5in]{Figures/MX}
%\caption{Radial velocity map of Hubble X in H$\alpha$.}
%\label{fig:MHX}
%\end{figure}

%\begin{figure}
%\centering 
%\includegraphics[width=2.5in]{Figures/M30D}
%\caption{Radial velocity map of 30 Dor in H$\alpha$ \citep{Castro:2018a}.}
%\label{fig:M30Dor}
%\end{figure}

%\begin{figure}
%\centering 
%\includegraphics[width=2.5in]{Figures/Car}
%\caption{Radial velocity map of Carina in H$\alpha$ taken from \citet{Damiani:2016a}.}
%\label{fig:MCar}
%\end{figure}

%\begin{figure}
%\centering 
%\includegraphics[width=3in]{Figures/M8H.png}
%\includegraphics[width=3.5in]{Figures/M8}
%\caption{Radial velocity map of M8 in H$\alpha$. Taken from \citep{Damiani:2017b} and \citet{1987A&A...176..338H}.}
%\label{fig:MM8}
%\end{figure}

%\begin{figure}
%\centering 
%\includegraphics[width=3in]{Figures/OrionL}
%\caption{Radial velocity map of Orion in H$\alpha$ taken from \citet{1987A&A...176..347H}.}
%\label{fig:MOrion}
%\end{figure}

%%%%%%%%%%%%%%%%%%%%%%%%%%%%%%%%%%%%%%%%%%%%%%%%%%
%%%%%%%%%%%%%%%%%%%%% END %%%%%%%%%%%%%%%%%%%%%%%%
%%%%%%%%%%%%%%%%%%%%%%%%%%%%%%%%%%%%%%%%%%%%%%%%%%

% Don't change these lines
\bsp	% typesetting comment
\label{lastpage}
\end{document}

% End of mnras_template.tex
%%% Local Variables:
%%% mode: latex
%%% TeX-master: t
%%% End:
