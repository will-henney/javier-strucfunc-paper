% mnras_template.tex 
%
% LaTeX template for creating an MNRAS paper
%
% v3.0 released 14 May 2015
% (version numbers match those of mnras.cls)
%
% Copyright (C) Royal Astronomical Society 2015
% Authors:
% Keith T. Smith (Royal Astronomical Society)

% Change log
%
% v3.0 May 2015
%    Renamed to match the new package name
%    Version number matches mnras.cls
%    A few minor tweaksto wording
% v1.0 September 2013
%    Beta testing only - never publicly released
%    First version: a simple (ish) template for creating an MNRAS paper

%%%%%%%%%%%%%%%%%%%%%%%%%%%%%%%%%%%%%%%%%%%%%%%%%%
% Basic setup. Most papers should leave these options alone.
\documentclass[fleqn,usenatbib, useAMS, a4paper, onecolumn]{mnras}


\usepackage{savesym}
\savesymbol{tablenum}
\usepackage{siunitx}
\restoresymbol{SIX}{tablenum}


% MNRAS is set in Times font. If you don't have this installed (most LaTeX
% installations will e fine) or prefer the old Computer Modern fonts, comment
% out the following line
\usepackage{newtxtext}
\usepackage[varg,varvw,smallerops]{newtxmath}
% Depending on your LaTeX fonts installation, you might get better results with one of these:
%\usepackage{mathptmx}
%\usepackage{txfonts}

% Use vector fonts, so it zooms properly in on-screen viewing software
% Don't change these lines unless you know what you are doing
\usepackage[T1]{fontenc}
\usepackage{ae,aecompl}


%%%%% AUTHORS - PLACE YOUR OWN PACKAGES HERE %%%%%

% Only include extra packages if you really need them. Common packages are:
\usepackage{graphicx}	% Including figure files
\let\Bbbk\relax
\usepackage{amsmath}	% Advanced maths commands
\usepackage{amssymb}	% Extra maths symbols
\usepackage{multicol}
\usepackage{enumerate}          % Better lists
\usepackage{xcolor}

\usepackage[spanish,es-minimal,english]{babel}

%% Package set-up
\usepackage{booktabs}
\usepackage{array}   % for \newcolumntype macro
\newcolumntype{L}{>{$}l<{$}} % math-mode version of lrc column types
\newcolumntype{R}{>{$}r<{$}} 
\newcolumntype{C}{>{$}c<{$}}

% Use more muted colors for links
\hypersetup{colorlinks=True, linkcolor=blue!50!black, citecolor=black,
  urlcolor=blue!50!black}

% Tweaks to siunitx configuration
\sisetup{
  % explicit""+" is useful for velocities
  retain-explicit-plus = true,
  % prefer 10^6 over 1 x 10^6
  retain-unity-mantissa = false,
  % Use x +/- e instead of x(e)  
  separate-uncertainty = true,
  % Make sure to pick up bold font when used in section heading for instance
  detect-weight = true,
}

%\usepackage{orcidlink}

\usepackage{upgreek}
%%%%%%%%%%%%%%%%%%%%%%%%%%%%%%%%%%%%%%%%%%%%%%%%%%

%%%%% AUTHORS - PLACE YOUR OWN COMMANDS HERE %%%%%

% Please keep new commands to a minimum, and use \newcommand not \def to avoid
% overwriting existing commands. Example:
%\newcommand{\pcm}{\,cm$^{-2}$}	% per cm-squared

% A better \ion command that works in more circumstances
\newcommand\ION[2]{#1\,\scalebox{0.9}[0.8]{\uppercase{#2}}}

\newcounter{ionstage}
\renewcommand{\ion}[2]{\setcounter{ionstage}{#2}% 
  \ensuremath{\mathrm{#1\,\scriptstyle\Roman{ionstage}}}}
  
\newcommand\hii{\ion{H}{2}}
\newcommand\pos{\ensuremath{_{\mathrm{pos}}}}
\newcommand\los{\ensuremath{_{\mathrm{los}}}}
\newcommand\noise{\ensuremath{_{\text{noise}}}}
\newcommand\obs{\ensuremath{_{\mathrm{obs}}}}
\newcommand\model{\ensuremath{_{\mathrm{mod}}}}
\newcommand\halpha{H${\alpha}$}
\newcommand\n{[\ion{N}{II}]$\lambda$6584}
\newcommand\oi{[\ion{O}{III}]$\lambda$5007}
\newcommand\s{[\ion{S}{II}]$\lambda$6737}
\newcommand\kms{$^{-1}$}

\newcommand\ha{\ensuremath{\text{H}\upalpha}}
\newcommand\hb{\ensuremath{\text{H}\upbeta}}
\newcommand\Wav[1]{\ensuremath{\lambda #1}}
% Chemical formulae
\newcommand*\chem[1]{\ensuremath{\mathrm{#1}}}
\newcommand\csound{\ensuremath{c_{\text{s}}}}
\newcommand\FNa{\textsuperscript{a}}
\newcommand\Mach{\ensuremath{\mathcal{M}}}
\newcommand\longsig[1]{\ensuremath{\langle \delta #1^2 \rangle^{1/2} / #1_0}}
\newcommand\shortsig[1]{\ensuremath{\sigma_{#1/#1_0}}}

%%%%%%%%%%%%%%%%%%%%%%%%%%%%%%%%%%%%%%%%%%%%%%%%%%
\newlength\SFwidth
\setlength\SFwidth{0.45\textwidth}
\newcommand\SFtwograph[2]{%
  \includegraphics[width=\SFwidth]{Figures/sf-emcee-#1}
  &  \includegraphics[width=\SFwidth]{Figures/sf-emcee-#2}
}
\newcommand\SFtwocorner[2]{%
  \includegraphics[width=\SFwidth]{Figures/corner-emcee-#1}
  &  \includegraphics[width=\SFwidth]{Figures/corner-emcee-#2}
}


\newcommand\sffigg[2]{%
  \begin{tabular}{@{}ll@{}}
    (a)& (b)\\
    \SFtwograph{#1}{#2}
  \end{tabular}%
}
\newcommand\sffigggg[4]{%
  \begin{tabular}{@{}ll@{}}
    (a)& (b)\\
    \SFtwograph{#1}{#2}\\
    (c)& (d)\\
    \SFtwograph{#3}{#4}\\
  \end{tabular}%
}
\newcommand\sfcfig[1]{%
  \includegraphics[width=\SFwidth]{Figures/corner-emcee-#1}%
}  
\newcommand\sfcfigg[2]{%
  \begin{tabular}{@{}ll@{}}
    (a)& (b)\\
    \SFtwocorner{#1}{#2}
  \end{tabular}%
}
\newcommand\sfcfigggg[4]{%
  \begin{tabular}{@{}ll@{}}
    (a)& (b)\\
    \SFtwocorner{#1}{#2}\\
    (c)& (d)\\
    \SFtwocorner{#3}{#4}\\
  \end{tabular}%
}
%%%%%%%%%%%%%%%%%%% TITLE PAGE %%%%%%%%%%%%%%%%%%%

% Title of the paper, and the short title which is used in the headers.
% Keep the title short and informative.
\title[Turbulence in H II regions]{SUPPLEMENTARY MATERIAL\\
  Turbulence in compact to giant H~II regions}

% The list of authors, and the short list which is used in the headers.
% If you need two or more lines of authors, add an extra line using \newauthor
\author[J. García-Vázquez et al.]{
  J. García Vázquez$^{1}$\thanks{
    E-mail: \href{mailto:jgarciav1600@alumno.ipn.mx}{jgarciav1600@alumno.ipn.mx}
  },
  William J. Henney$^{2}$
  and H. O. Castañeda$^{1}$\thanks{Deceased.}
\\
% List of institutions
$^{1}$Escuela Superior de Física y Matemáticas, Instituto Politécnico Nacional, U.P. Adolfo López Mateos, Zacatenco, Ciudad de México, México C.P. 07738\\
$^{2}$Instituto de Radioastronomía y Astrofísica,
Universidad Nacional Autónoma de México,
Antigua Carretera a Pátzcuaro \#8701,\\
Ex-Hda. San José de la Huerta, 
Morelia, Michoacán, México C.P. 58089\\
}

% These dates will be filled out by the publisher
\date{Accepted XXX. Received YYY; in original form ZZZ}

% Enter the current year, for the copyright statements etc.
\pubyear{2023}

% Don't change these lines
\begin{document}
\label{firstpage}
\pagerange{\pageref{firstpage}--\pageref{lastpage}}
\maketitle

% % Abstract of the paper
% \begin{abstract}
%   Appendix D.
% \end{abstract}

% % Select between one and six entries from the list of approved keywords.
% % Don't make up new ones.
% \begin{keywords}
% HII regions -- ISM: kinematics and dynamics -- turbulence 
% \end{keywords}


\definecolor{WillCommentColor}{rgb}{0.6,0.11,0.4}
\newcommand\WILL[1]{\textbf{\color{WillCommentColor}#1}}
%%%%%%%%%%%%%%%%%%%%%%%%%%%%%%%%%%%%%%%%%%%%%%%%%%

\appendix
\addtocounter{section}{3}


\section{Additional covariance corner plots for model fits}
\label{sec:addit-covar-corn}
Supplementary online-only material.
%\clearpage

\begin{figure*}
  \centering
  \sfcfigggg{OrionS}{OrionLH}{M8}{CarC}
  \caption{
    Corner plots of covariances between
    fitted model parameters
    of \ha{} structure function
    for Galactic \hii{} regions.
  }
  \label{fig:corner-Galactic}
\end{figure*}


\begin{figure*}
  \centering
  \sfcfigg{N346}{Dor}
  \caption{
    Same as figure~\ref{fig:corner-Galactic}
    except for Magellanic Cloud \hii{} regions. 
  }
  \label{fig:corner-MC}
\end{figure*}

\begin{figure*}
  \centering
  \sfcfigggg{HV}{HX}{N595}{N604H}
  \caption{
    Same as figure~\ref{fig:corner-Galactic}
    except for \hii{} regions in more distant
    Local Group Galaxies.
  }
  \label{fig:corner-ExtraGal}
\end{figure*}

\begin{table*}
\begin{center}\caption{Fitting parameters.}
\begin{tabular}{cCCCCCCCC}\toprule
\hii{}    &  \text{No. points} & \text{Relatively}  & \text{Weight / No. points}     &  \text{Weight / No. points}     & \text{Max.}  & \text{Reduced} & \text{Autocorrelation } & \text{Acceptance}\\
Region    &  \text{merged} & \text{uncertainty} & \text{Small scales} &  \text{Large scales} & \text{separation}  & \text{chi-square} & \text{time} & \text{fraction}\\
\midrule
Orion     & 3 & 0.02    &  -           &   -        & 0.5L  & 0.82 & 60  & 0.55\\
EON       & - & 0.03   &  2.0 / 3     &   3.0 / 3        & 0.7L  & 0.62 & 40 & 0.64\\
Lagoon    & 2 & 0.08    &  2.5 / 16    &   2.5 / 8  & 0.5L  & 0.94 & 72  & 0.50\\
Carina    & 2 & 0.055   &  3.0 / 14    &   3.0 / 4  & 0.5L  & 0.94 & 72  & 0.50\\
30 Dor    & 5 &  0.07   &  -           &   -        & 0.9L  & 0.98 & 84  & 0.54\\
NGC 346   & 3 & 0.02    &  -           &   -        & 0.5L  & 0.83 & 64  & 0.54\\
Hubble V  & - & 0.05    &  2.0 / 3     &   -        & 0.6L  & 0.86 & 340 & 0.44\\
Hubble X  & - & 0.032   &  2.0 / 4     &   2.0 / 6  & 0.5L  & 0.73 & 79  & 0.49\\
NGC 595   & - &  0.055  &  4.5 / 3     &   3.0 / 5  & 0.5L  & 0.83 & 67  & 0.51\\
NGC 604   & - &  0.08  &   2.0 / 3    &   1.5 / 2 & 0.5L  & 0.92 & 112 & 0.45\\
\bottomrule
\end{tabular}\label{tab:fitting-parameters}
\end{center}
\end{table*} 

%%%%%%%%%%%%%%%%%%%%%%%%%%%%%%%%%%%%%%%%%%%%%%%%%%
%%%%%%%%%%%%%%%%%%%%% END %%%%%%%%%%%%%%%%%%%%%%%%
%%%%%%%%%%%%%%%%%%%%%%%%%%%%%%%%%%%%%%%%%%%%%%%%%%

%%% Local Variables:
%%% mode: latex
%%% TeX-master: "supplementary_material_strucfunc_paper"
%%% End:

% Don't change these lines
\bsp	% typesetting comment
\label{lastpage}
\clearpage
\end{document}

% End of mnras_template.tex
%%% Local Variables:
%%% mode: latex
%%% TeX-master: t
%%% End:
